


% Dies ist ein Kommentar

\documentclass{article}
%
% Praeambel
%
\usepackage[ngerman]{babel}
\usepackage{amsmath}

\begin{document}
\begin{equation}
    n! = \prod_{i=1}^n i
\end{equation}
\begin{align}
    f(n) &= f(n-1) + f(n-2)\\
    \sum\limits_{k=1}^n k &= 1 + 2 + \dots + n \nonumber \\
        &= \frac{n(n+1)}{2}
\end{align}
\end{document}
