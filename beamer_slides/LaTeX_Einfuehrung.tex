\documentclass[t]{beamer}

\usepackage[T1]{fontenc}
\usepackage[utf8]{inputenc}

\newcommand{\titel}{Einf\"uhrung in \LaTeX}
\newcommand{\titelkurz}{Einf\"uhrung in \LaTeX}
\newcommand{\vortragender}{Raphael M\"unster}
\newcommand{\koautoren}{}
\newcommand{\ortdatum}{M\"arz 2017}



\usepackage{array}
\usepackage{ragged2e}

\usepackage[ngerman,english]{babel}
\usepackage[babel,german=quotes]{csquotes}
\usepackage{graphicx}
\usepackage{tikz}
\usetikzlibrary{shapes.arrows,calc,decorations.pathmorphing}
%%\usepackage{gitinfo}
\usepackage{shadowtext}

\usetheme{-tud-4-3}

% Title page
\title[\titelkurz]{\titel}
\author{Raphael M\"unster, Folien entliehen mit freundlicher Genehmigung von Christopher Basting}
\date{\ortdatum\\[-0.2cm]}
\institute{https://depot.tu-dortmund.de/xlu8z}

% ------------------------------------------------------------------------------------
% VORTRAG ESCO 2012     FRIEDRICH-ALEXANDER-UNIVERSITAET ERLANGEN-NUERNBERG
% ------------------------------------------------------------------------------------
%     erstellt von Christopher Basting, 21. Juni 2012
% ------------------------------------------------------------------------------------
% abbrev.tex
% ------------------------------------------------------------------------------------

% Mathe spezifische Packages
\usepackage{amsmath,amsfonts}
\usepackage{dsfont}

\renewcommand{\d}{{\rm d}} % d fuer dx bei Ableitungen / Integralen
\newcommand{\gdw}{\Leftrightarrow} % <=>
\newcommand{\x}{\cdot}
\newcommand{\ma}{\left( \begin{array}{*{19}{c}}} % Matrix oeffnen
\newcommand{\me}{\end{array}\right)} % Matrix schliessen
\newcommand{\mA}{\left[ \begin{array}{*{19}{c}}} % Matrix oeffnen
\newcommand{\mE}{\end{array}\right]} % Matrix schliessen
\newcommand{\C}{\mathds{C}}
\newcommand{\E}{\mathds{1}}
\newcommand{\F}{\mathds{F}}
\newcommand{\K}{\mathds{K}}
\newcommand{\N}{\mathds{N}}
\newcommand{\Q}{\mathds{Q}}
\newcommand{\R}{\mathds{R}}
\newcommand{\Z}{\mathds{Z}}
\newcommand{\id}{\mathds{1}}
\renewcommand{\d}{\,{\rm d}}
\newcommand{\D}{{\rm D}}
\newcommand{\richtig}{\checkmark}
\newcommand{\falsch}{\lightning}
\newcommand{\ra}{\rightarrow}
\newcommand{\Ra}{\Rightarrow}
\newcommand{\la}{\leftarrow}
\newcommand{\La}{\Leftarrow}

\newcommand{\abbildung}[3]{#1 \, : \, #2 \rightarrow #3}
\newcommand{\grad}{\nabla}

%\renewcommand{\phi}{\varphi}
\newcommand{\cL}{\mathcal{L}}
\newcommand{\cO}{\mathcal{O}}

\newcommand{\eps}{\varepsilon}

\renewcommand{\vec}{\bold}
\newcommand{\dist}{{\rm dist}}

\newcommand{\mat}[1]{\textsf{\textbf{#1}}}

\usepackage{amscd,amssymb} 

\providecommand{\abs}[1]{\lvert#1\rvert}
\providecommand{\norm}[1]{\left\lVert#1\right\rVert}
\providecommand{\normNoAdjust}[1]{\lVert#1\rVert}
\providecommand{\transpose}[1]{{#1}^\top}
\providecommand{\inversetranspose}[1]{{#1}^{-\top}}



\shadowoffsetx{0.7pt}
\shadowoffsety{0.7pt}


\newcommand{\itemra}{\item[$\rightarrow$]}
%\newcommand{\TUgreen}{\color{TUgreen}}
\newcommand{\TUgreen}{\color{red}}
\usepackage{hyperref}

\usepackage{listings}

\lstset{breaklines=true,      % sets automatic line breaking
basicstyle=\ttfamily\LARGE,
language=TeX,
escapechar=@
}

\usepackage{fancybox, graphicx}

\usepackage{lmodern}

% Einige Befehle (s. Thomas Rohkaemper's Slides)
\definecolor{darkgray}{gray}{0.3}
\newcommand{\kommandozeile}[1]{{{\ttfamily\bfseries \color{white}\colorbox{black}{#1}}}}
\newcommand{\dateiname}[1]{{\ttfamily\bfseries \color{red}#1}}
\newcommand{\emphword}[1]{{\color{red}#1}}
\newcommand{\keyword}[1]{\texttt{\color{blue}#1}}
\newcommand{\emphkeyword}[1]{\texttt{\color{red}#1}}
\newcommand{\befehl}[1]{\texttt{\color{blue}\symbol{`\\}#1}}
\newcommand{\tabitem}{~~\llap{\color{red}\textbullet}~~}

\newcommand{\latexBeispielDatei}[2]{
	\begin{columns}[T]
		\begin{column}{0.54\textwidth}
			\begin{block}{#1}
				\lstinputlisting{#2.tex}
			\end{block}
		\end{column}
		\begin{column}{0.44\textwidth} \vspace{-2.5cm}
			\shadowbox{\includegraphics[width=0.9\textwidth]{#2.pdf}}
		\end{column}
	\end{columns}
}

\newcommand{\latexBeispielDateiCodeAnmerkungen}[3]{
	\begin{columns}[T]
		\begin{column}{0.54\textwidth}
			\begin{block}{#1}
				\lstinputlisting{#2.tex}
			\end{block}
		\end{column}
		\begin{column}{0.44\textwidth}
			#3
		\end{column}
	\end{columns}
}

\newcommand{\latexBeispielDateiKlein}[2]{
	\begin{columns}[T]
		\begin{column}{0.54\textwidth}
			\begin{block}{#1}
				\lstinputlisting[basicstyle=\ttfamily\large]{#2.tex}
			\end{block}
		\end{column}
		\begin{column}{0.44\textwidth} \vspace{-2.5cm}
			\shadowbox{\includegraphics[width=0.9\textwidth]{#2.pdf}}
		\end{column}
	\end{columns}
}


\newcommand{\latexBeispielDirekt}[2]{
	\begin{columns}[c]
		\begin{column}{0.54\textwidth}
			\begin{block}{#1}
			\lstinputlisting{#2}
			\end{block}
		\end{column}
		\begin{column}{0.44\textwidth}
			\shadowbox{\begin{minipage}{0.99\textwidth} \input{#2} \end{minipage}}
		\end{column}
	\end{columns}
}

\newcommand{\latexBeispielDirektKlein}[2]{
	\begin{columns}[c]
		\begin{column}{0.54\textwidth}
			\begin{block}{#1}
			\lstinputlisting[basicstyle=\ttfamily\large]{#2}
			\end{block}
		\end{column}
		\begin{column}{0.44\textwidth}
			\shadowbox{\begin{minipage}{0.99\textwidth} \input{#2} \end{minipage}}
		\end{column}
	\end{columns}
}

\newcommand{\latexBeispielDirektFake}[3]{
	\begin{columns}[c]
		\begin{column}{0.54\textwidth}
			\begin{block}{#1}
			\lstinputlisting{#2}
			\end{block}
		\end{column}
		\begin{column}{0.44\textwidth}
			\shadowbox{\begin{minipage}{0.99\textwidth} #3 \end{minipage}}
		\end{column}
	\end{columns}
}

\begin{document}

%
%
%%%%%%%%%%%%%%%%%%%%%%%%%%%%%%%%%%%%%%%%%%%%%%%%%%%%%%%%%%%%%%%%%%%%%%%%%%%%%
%
%
% Kopf- und Fußzeile abschalten
\frame[plain,c]{\titlepage}

% ab jetzt Autoren ohne Unterstreichung des Vortragenden
\author{\vortragender}

% keine eigene Seite für jede Section einblenden (kann mit \activateOwnSectionPage rückgängig gemacht werden)
\deactivateOwnSectionPage 


%%%%%%%%%%%%%%%%%%%%%%%%%%%%%%%%%%%%%%%%%%%%%%%%%%%%%%%%%%%%%%%%%%%%%%%%%%%%%
% N O R M A L E   U N I - V E R S I O N
%%%%%%%%%%%%%%%%%%%%%%%%%%%%%%%%%%%%%%%%%%%%%%%%%%%%%%%%%%%%%%%%%%%%%%%%%%%%%

\begin{frame}
	\frametitle{Über \LaTeX}

	\begin{itemize}
		\itemra ein Textsatzsystem
        \itemra \TeX{} ist eine Programmiersprache von Donald E. Knuth
        \itemra \LaTeX{} ist eine Sammlung von Makros für \TeX{} von Leslie Lamport
		\itemra Dokumente werden aus Quelltext erzeugt
		\itemra Quelltext ist ``logisches Markup'' (wie HTML)
		\itemra kein WYSIWYG (\textbf What\textbf You\textbf See\textbf Is\textbf What\textbf You\textbf Get)
		\itemra standardisierte Dokumente
		\itemra einfacher Formelsatz
		\itemra kostenlos und für alle gängigen Betriebssysteme erhältlich
		\itemra Standard für wissenschaftliche Veröffentlichungen
	\end{itemize}
\end{frame}

\begin{frame}
	\frametitle{Woher bekommt man \LaTeX?}
	
	\begin{itemize}
		\item offizielle Homepage: \url{http://www.latex-project.org}
		\item verschiedene Distributionen verfügbar:
		\begin{center}
			\begin{tabular}{rl}
				{\TUgreen TeX Live} & \url{http://tug.org/texlive/} \\
					& Installer oder direkt von USB-Stick startbar\\
				{\TUgreen Mik TeX} & \url{http://www.miktex.org} \\
					& für Windows-Systeme; Installer oder portable Version\\
				{\TUgreen Mac TeX} & \url{http://tug.org/mactex/} \\
					& TeX Live für Apple Mac-Systeme
			\end{tabular}
		\end{center}
		\item detailierte Beschreibung des Installationsvorgangs für \TeX Live:
		\begin{center}
			\url{http://www.dante.de/tex.html}
		\end{center}
	\end{itemize}
\end{frame}

\begin{frame}
	\frametitle{Grundaufbau einer \LaTeX-Datei}
	
	\begin{itemize}
		\item Präambel
			\begin{itemize}
				\item[$\to$] Stilvorgaben (Schriftart, Schriftgröße, Seitenränder, Kopf-/Fußzeile, usw.)
				\item[$\to$] Pakete mit zusätzlichen Funktionen laden (z. B. Farben, Verlinkung, usw.)\\~
			\end{itemize}
		\item Dokument
			\begin{itemize}
				\item[$\to$] Inhaltsverzeichnis
				\item[$\to$] Kapitel und Überschriften
				\item[$\to$] Text und Bilder
				\item[$\Ra$] der gesamte Inhalt
			\end{itemize}
	\end{itemize}
\end{frame}

\begin{frame}[fragile]
	\frametitle{\LaTeX: Ablauf der Dokumenterstellung}

\begin{center}
	\begin{tikzpicture}%[nodes=draw]
	\node[outer sep=5pt, anchor=base]            (a) {~};
	\node[outer sep=5pt, anchor=base] at (5,0) (b) {~};
	\node[outer sep=5pt, anchor=base] at (7,0) (c) {~};
	\node[outer sep=5pt, anchor=base] at (12,0) (d) {~};
	\node[outer sep=5pt, anchor=base] at (14,0) (e) {~};
	\node[outer sep=5pt, anchor=base] at (19,0) (f) {~};
	\path let \p0=($(a.west)-(b.east)$) in
	  (a) --
	  node [
	    pos=.5,
	    auto=false,
	    shape=single arrow,
	    anchor=tail,
	    draw=none,
	%    line join=round,
	    xshift=-veclen(\p0)/2,
	    minimum height=veclen(\p0)-\pgfkeysvalueof{/pgf/outer xsep},
	    fill=TUgreen,
	    decorate, decoration={
	      name=random steps,
	      segment length=+.5pt,
	      amplitude=.5pt}
	  ] {Editor}
	  (b);
\node[align=left] at (2,-2) {zur Eingabe des\\ sog. \textbf{Quellcode}\\(bei uns: Kile)};	  
	  
	\path let \p0=($(c.west)-(d.east)$) in
	  (c) --
	  node [
	    pos=.5,
	    auto=false,
	    shape=single arrow,
	    anchor=tail,
	    draw=none,
	%    line join=round,
	    xshift=-veclen(\p0)/2,
	    minimum height=veclen(\p0)-\pgfkeysvalueof{/pgf/outer xsep},
	    fill=TUgreen,
	    decorate, decoration={
	      name=random steps,
	      segment length=+.5pt,
	      amplitude=.5pt}
	  ] {\LaTeX-Compiler}
	  (d);
\node[align=left] at (9.3,-2) {übersetzt den Quellcode\\ in ein Dokumentformat\\ (z. B. PDF)};	 

	\path let \p0=($(e.west)-(f.east)$) in
	  (e) --
	  node [
	    pos=.5,
	    auto=false,
	    shape=single arrow,
	    anchor=tail,
	    draw=none,
	%    line join=round,
	    xshift=-veclen(\p0)/2,
	    minimum height=veclen(\p0)-\pgfkeysvalueof{/pgf/outer xsep},
	    fill=TUgreen,
	    decorate, decoration={
	      name=random steps,
	      segment length=+.5pt,
	      amplitude=.5pt}
	  ] {Betrachter}
	  (f);
\node[align=left] at (16.6,-2) {zeigt das Dokument an};
	\end{tikzpicture}
\end{center}

\end{frame}



\begin{frame}
	\frametitle{Kile - \LaTeX Entwicklungsumgebung}
	
	\begin{itemize}
			\item erleichtert das Bearbeiten von mehreren Dokumenten
			\item Autovervollständigung
			\item Hilfe zum Kompilieren der Dokumente
			\item Syntaxhervorhebung
			\item Rechtschreibprüfung
			\item Code-Faltung
	\end{itemize}
\end{frame}

\begin{frame}[plain]
	\includegraphics[width=\textwidth]{screenshots/kile-start.png}
\end{frame}

\begin{frame}[plain,noframenumbering]
	\includegraphics[width=\textwidth]{screenshots/kile-new.png}
\end{frame}

\begin{frame}[plain,noframenumbering]
	\includegraphics[width=\textwidth]{screenshots/kile-new-doc.png}
\end{frame}

\begin{frame}[plain,noframenumbering]
	\includegraphics[width=\textwidth]{screenshots/kile-new-save-as.png}
\end{frame}

\begin{frame}[plain,noframenumbering]
	\includegraphics[width=\textwidth]{screenshots/kile-new-save-as-create-folder.png}
\end{frame}

\begin{frame}[plain,noframenumbering]
	\includegraphics[width=\textwidth]{screenshots/kile-new-save-as-2.png}
\end{frame}

\begin{frame}[plain,noframenumbering]
	\includegraphics[width=\textwidth]{screenshots/kile-quickbuild.png}
\end{frame}

\begin{frame}[plain,noframenumbering]
	\includegraphics[width=\textwidth]{screenshots/kile-quickbuild-2.png}
\end{frame}

\begin{frame}[fragile]
	\frametitle{Hello World!}
	
	\latexBeispielDatei{Ein erstes \LaTeX-Dokument}{examples/hello_world/hello_world}
\end{frame}

\begin{frame}[fragile]
	\frametitle{Generierung einer PDF-Datei}
    Um eine PDF-Datei aus einem \LaTeX{} Dokument \dateiname{datei.tex} zu erzeugen, muss der \LaTeX-Compiler aufgerufen werden:\\
	\kommandozeile{pdflatex datei.tex} \\[1cm]
Alternativ kann man wie folgt vorgehen:
	\begin{enumerate}
		\item \kommandozeile{latex datei.tex}\\
		Dieser Befehl ruft den \LaTeX-Compiler auf und erzeugt die DVI-Datei \dateiname{datei.dvi}.
		\item \kommandozeile{dvips datei.dvi}\\
		Konvertiert die \dateiname{datei.dvi} in eine Post-Script Datei \dateiname{datei.ps}.
		\item \kommandozeile{ps2pdf} oder direkt \kommandozeile{dvipdf}\\
		Erzeugt eine PDF-Datei \dateiname{datei.pdf}.
	\end{enumerate}
\end{frame}

\begin{frame}
	\frametitle{Eingabe eines einfachen Textes}
	
	\latexBeispielDateiKlein{Quelltext}{examples/Eingabe_Quelltext/Eingabe_Quelltext}
\end{frame}

\begin{frame}
	\frametitle{Befehle und Umgebungen}
	\begin{itemize}
        \item \emphword{Befehle} haben in \LaTeX{} die Form \befehl{befehl}
		\item Beispiel: \befehl{textit\{kursiv\}} schaltet auf {\it Kursivschrift} um
		\item Manche Befehle benötigen zusätzliche Parameter, z. B.\\
		\befehl{textbf\{Text, der fett gedruckt sein soll\}}\\
		oder optionale Parameter, z. B.\\
		\befehl{documentclass[a4paper]\{article\}}\\[1cm]
		\item \emphword{Umgebungen} beeinflussen den gesamten enthaltenen Text:\\
		\befehl{begin\{center\}} \\
		~~~~zentrierter Text\\
		\befehl{end\{center\}}\\
		Auch hier ist die Angabe von Parametern möglich.
	\end{itemize}
\end{frame}

\begin{frame}
	\begin{block}{\huge Aufgabe 1 (Hello World)}
		Erstellen Sie folgendes "Hello-World" Beispiel.
\begin{verbatim}
\documentclass{article}
\begin{document}
    Hello World!
\end{document}
\end{verbatim}
	\end{block}
\end{frame}
\begin{frame}
	\frametitle{Spezielle Zeichen}
	\begin{itemize}
		\item bei Verwendung des ASCII-Zeichensatzes: Buchstaben ohne Umlaute, Zahlen und einige Sonderzeichen
		\item manche Zeichen sind \LaTeX-Steuerzeichen und daher reserviert (\$, \_, \{, \},\textbackslash )
        \item solche Sonderzeichen können durch \textbackslash{} maskiert werden: \\[0.5cm]
		\begin{center}
			\begin{tabular}{|cc|cc|} \hline
				\$ & \textbackslash\$ & \% & \textbackslash\% \\
				\{ & \textbackslash\{ & \} & \textbackslash\} \\
				\# & \textbackslash\# & \_ & \textbackslash\_ \\
				\& & \textbackslash\& & & \\ \hline
			\end{tabular}
		\end{center}
	\end{itemize}
\end{frame}

\begin{frame}[fragile]
	\frametitle{Deutsche Texte - Sprachpakete}
    Ohne weitere Angaben nimmt \LaTeX{} an, dass der eingegebene Text in englischer Sprache ist. Daher muss ggf. ein zusätzliches Sprachpaket eingebunden werden:
	\begin{center}
		\begin{block}{Beispiel-Header: deutsches Sprachpaket}
			\begin{lstlisting}
\documentclass[a4paper]{article}      % DIN-A4 Papierformat
\usepackage[ngerman]{babel}           % deutsche Benennung
\usepackage[utf8]{inputenc}
\begin{document}
  ...
\end{document}
			\end{lstlisting}
		\end{block}
	\end{center} \vspace{-1cm}
	\begin{itemize}
		\item \keyword{babel} sorgt für Unterstützung anderer Sprachen (Formate, Umlaute, Benennungen, Silbentrennung)
		\item \keyword{inputenc} unterstützt die direkte Eingabe von Zeichen über die Tastatur
	\end{itemize}
\end{frame}


\begin{frame}[fragile]
	\frametitle{Deutsche Texte -- Umlaute und Anführungszeichen}
	
	Eingabe deutscher Texte:
	\begin{itemize}
		\item Umlaute: \befehl{"a}, \befehl{"o}, \befehl{"u}, \befehl{ss} für ä, ö, ü, ß
		\item Anführungszeichen: \befehl{glq}, \befehl{grq} bzw. \befehl{glqq}, \befehl{grqq} für \glq einfache\grq~ bzw. \glqq doppelte\grqq~ Anführungszeichen
	\end{itemize}
	\vfill
	
	\latexBeispielDirekt{Beispiel: deutsche Umlaute}{examples/Deutsche_Umlaute/Deutsche_Umlaute}
	\vfill
\end{frame}

\begin{frame}[fragile]
	\frametitle{Deutsche Texte - Silbentrennung}
	\begin{itemize}
		\item erfolgt automatisch
		\item mögliche Trennstellen können  durch \befehl{-} auch angegeben werden, z. B.\\
		\lstinline$Donau\-dampf\-schiff\-fahrts\-gesell\-schaft$\\
		oder für das gesamte Dokument in der Präambel:
		\lstinline$\hypenation{Donau\-dampf\-schiff\-fahrts\-gesell\-schaft}$
	\end{itemize}

\end{frame}

\begin{frame}[fragile]
	\frametitle{Textformatierung -- Schriftstil}
	
	\begin{center}
		\begin{tabular}{l|ll|l}
			Familie & \multicolumn{2}{c|}{Befehle} & Beispiel \\ \hline
			normal (mit Serifen) & \befehl{rmfamily} & \befehl{textr} & \textrm{normal} \\
			serifenfrei & \befehl{sffamily} & \befehl{textsf} & \textsf{serifenfrei} \\
			Schreibmaschine & \befehl{ttfamily} & \befehl{texttt} & \texttt{Schreibmaschine} \\
		\multicolumn{4}{c}{~} \\
			Varianten & \multicolumn{2}{c|}{Befehle} & Beispiel \\ \hline
			aufrecht & \befehl{upshape} & \befehl{textup} & \textup{aufrecht} \\
			italic & \befehl{itshape} & \befehl{textit} & \textit{italic}\\
			Kapitälchen & \befehl{scshape} & \befehl{textsc} & \textsc{Kapitälchen}\\
			fett & \befehl{bfseries} & \befehl{textbf} & \textbf{fett}\\
			unterstrichen & ~ & \befehl{underline} & \underline{unterstrichen}
		\end{tabular}
	\end{center}
\end{frame}

\begin{frame}[fragile]
	\frametitle{Textformatierung -- Schriftgröße}
	\begin{center}
		\begin{tabular}{ll}
		\befehl{tiny} & \tiny{winzig} \\
		\befehl{small} & \small{klein} \\
		\befehl{footnotesize} & {\fontsize{14}{14}\selectfont{}Fußnotengröße}\\
		\befehl{normalsize} & normale Größe\\
		\befehl{large} & {\fontsize{25}{25}\selectfont{}groß} \\
		\befehl{Large} & {\fontsize{30}{30}\selectfont{}größer} \\
		\befehl{huge} & {\fontsize{40}{40}\selectfont{}riesig} \\
		\befehl{Huge} & {\fontsize{50}{50}\selectfont{}Riesig}
		\end{tabular}
	\end{center}
	\vfill
	\begin{itemize}
		\item für einige Wörter: \lstinline${\huge riesig}$
		\item für ganze Absätze: \lstinline$\begin{tiny} ... \end{tiny}$
		\item alternativ: punktgenau durch \lstinline${\fontsize{40}{48}\selectfont{}Test}$\\
		das erste Argument gibt die Schriftgröße an, das zweite den Grundlinienabstand
	\end{itemize}
\end{frame}



\begin{frame}[fragile]
	\frametitle{Textformatierung -- minipages}
	\begin{itemize}
		\item die \keyword{minipage}-Umgebung wird genutzt, wenn in Boxen komplexere Inhalte  dargestellt werden sollen
		\item es muss die Breite angegeben werden
	\end{itemize}
	\latexBeispielDirekt{Beispiel: minipages}{examples/minipages/minipages.tex}
\end{frame}

\begin{frame}[fragile]
	\frametitle{Textformatierung -- Absatzausrichtung} \vspace{-0.7cm}
	\latexBeispielDirekt{Beispiel: Absatzausrichtung}{examples/absatzausrichtung/absatzausrichtung.tex}
	\vfill
\begin{itemize}
	\item Standard: Blocksatz
	\item alternativ kann man auch die Befehle \befehl{raggedleft}, \befehl{raggedright} und \befehl{centering} verwenden (statt den obigen Umgebungen)
\end{itemize}
\end{frame}

\begin{frame}[fragile]
	\frametitle{Textformatierung - Zitate und Gedichte}
	\latexBeispielDirektKlein{Beispiel: Zitate und Gedichte}{examples/Zitate_und_Gedichte/Zitate_und_Gedichte.tex}
\end{frame}

\begin{frame}[fragile]
	\frametitle{Text ohne Formatierung}
	\begin{itemize}
		\item um Programmcode oder Beispiele ohne Formatierung auszugeben, gibt es die \keyword{verbatim}-Umgebung
		\item innerhalb eines Textes kann man den Befehl \befehl{verb} nutzen; das erste Zeichen nach dem Befehl wird dann als Endmarker interpretiert
	\end{itemize}\vfill
	\latexBeispielDirekt{Beispiel: \keyword{verbatim}-Umgebung}{examples/verbatim/verbatim.tex}
\end{frame}

\begin{frame}[fragile]
	\frametitle{Randnotizen und Fußnoten}
	\latexBeispielDatei{Beispiel: Randnotiz und Fußnote}{examples/Randnotiz_und_Fussnote/Randnotiz_und_Fussnote}
\end{frame}
\begin{frame}[fragile]
	\frametitle{Dokumentenklasse}
	
	\begin{block}{Allgemeine Form}
		\lstinline$\documentclass[option1,option2]{klasse}$
	\end{block}
	\vfill
	einige verfügbare Klassen:
	\begin{itemize}
		\item \emphkeyword{article}, \emphkeyword{scrartcl} für Artikel, Ausarbeitungen oder Berichte
		\item \emphkeyword{report}, \emphkeyword{scrreprt} für Ausarbeitungen, Diplomarbeiten, Skripten oder Dissertationen
		\item \emphkeyword{book}, \emphkeyword{scrbook} für Bücher
		\item \emphkeyword{letter}, \emphkeyword{scrlttr2} für Briefe
	\end{itemize}
	\vfill
\end{frame}


\begin{frame}[fragile]
	\frametitle{Dokumentenklasse -- häufig benutzte Optionen}
	
	\begin{tabular}{rl}
		\emphkeyword{draft} & Entwurfsmodus (markiert überstehenden Text, zeigt Grafiken als Box) \\
		\emphkeyword{11pt} & Schriftgr\"o\ss{}e 11 Punkte \\
		\emphkeyword{12pt} & Schriftgr\"o\ss{}e 12 Punkte\\
		\emphkeyword{a4paper} & Gr\"o\ss{}en anpassen f\"ur Din A4\\
		\emphkeyword{titlepage} & Titelseite erzwingen f\"ur \keyword{\texttt{article}}\\
		\emphkeyword{twocolumn} & zweispaltiges Dokument\\
		\emphkeyword{twoside} & Vorder- und R\"uckseite der Seiten wird bedruckt\\
		\emphkeyword{fleqn} & Formeln linksb\"undig setzen\\
		\emphkeyword{leqno} & Nummerierung von Formeln links statt rechts
	\end{tabular}
\end{frame}


\begin{frame}
	\frametitle{Titelseite aus Metadaten}
	
	\only<1>{
	\latexBeispielDateiCodeAnmerkungen{automatisch generierte Titelseite}{examples/Titelseite/Titelseite}{
	\begin{tabular}{rp{6cm}}
		\keyword{\~} & unzerbrechliches Leer\-zeichen\\
		\befehl{title} & Titel des Dokuments \\
		\befehl{author} & Autoren (Schlüsselwort \befehl{and} bei Angabe mehrerer Autoren) \\
		\befehl{date} & Veröffentlichungsdatum\\
		\befehl{maketitle} & generiert die Titelseite aus den Metadaten
	\end{tabular}
	}}
	\only<2>{\latexBeispielDatei{automatisch generierte Titelseite}{examples/Titelseite/Titelseite}}
\end{frame}

\begin{frame}
	\frametitle{Eigene Titelseiten}
	\begin{itemize}
		\item um die Titelseite selbst zu gestalten, kann man die \emphkeyword{titlepage}-Umgebung nutzen
	\end{itemize} \vfill
		\latexBeispielDirektFake{Beispiel: eigene Titelseite}{examples/Eigene_Titelseite/Eigene_Titelseite}{\textbf{\Huge Nicht wichtig}\\geschrieben von mir\ldots} \vfill
	\begin{itemize}
		\item die Titelseite taucht an der Stelle im Text auf, an der sie definiert wird
		\item bei einigen Dokumentenklassen (z. B. \emphword{book}) nimmt die Titelseite auch eine ganze Seite ein
	\end{itemize}
\end{frame}

\begin{frame}[fragile]
	\frametitle{Abstract (Zusammenfassung)}
	\begin{itemize}
		\item durch die \emphkeyword{abstract}-Umgebung kann man eine Zusammenfassung des Inhalts angeben
	\end{itemize} \vfill
	\latexBeispielDirektFake{Beispiel: Abstract (Zusammenfassung)}{examples/abstract/abstract}{
	\begin{center} \textbf{\LARGE Abstract} \end{center} \noindent \ldots Zusammenfassung des Textes \ldots} \vfill
	\begin{itemize}
        \item der angezeigte Name \glqq Abstract\grqq{} kann individualisiert werden:\\
		\lstinline$\renewcommand{\abstractname}{MeinName}$
	\end{itemize}
\end{frame}

\begin{frame}
	\frametitle{Strukturierung des Dokuments}
	
	\begin{itemize}
		\item Gliederung in Kaptiel und Unterkapitel
		\begin{itemize}
			\item \befehl{part\{\}} (nicht in allen Dokumentklassen)
			\item \befehl{chapter\{\}} (nicht in allen Dokumentklassen)
			\item \befehl{section\{\}}
			\item \befehl{subsection\{\}}
			\item \befehl{subsubsection\{\}}
			\item \befehl{paragraph\{\}}
			\item \befehl{subparagraph\{\}}
		\end{itemize}
		\item in den geschweiften Klammern \{\} kann der Name des jeweiligen (Unter-)Kapitels angegeben werden
		\item mit dem Befehl \befehl{appendix} kann markiert werden, dass die nachfolgenden (Unter-)Kapitel zum Anhang gehören
	\end{itemize}
\end{frame}

\begin{frame}[fragile]
	\frametitle{Mehrere Quelldateien}
	\begin{itemize}
		\item um Quelltext aus einer anderen Datei einzufügen, wird der \befehl{input}-Befehl genutzt\\
		\lstinline$\input{datei}$
		\item um ganze Kapitel einzubinden, verwendet man stattdessen den Befehl\\
		\lstinline$\include{kapitel}$
		\item zur Gliederung des \LaTeX-Quellcodes schreibt man Kapitel in jeweils eigene Dateien\\
		\lstinline$\include{kapitel1}$\\
		\lstinline$\include{kapitel2}$\\
		usw.
	\end{itemize}
\end{frame}

\begin{frame}
	\vspace{-1cm}
	\begin{block}{\huge Aufgabe 2 (Ein erstes Dokument)}
		Erstellen Sie ein einfaches \LaTeX-Dokument. Teilen Sie den Text in Kapitel und Unterkapitel ein. Setzen Sie einige Wörter im Text kursiv, im Fettdruck oder in Schreibmaschinenschrift und in verschiedenen Schriftgrößen. Verwenden Sie zum Beispiel das Paket \keyword{blindtext} zur Erzeugung von Text-Passagen.
	\end{block}
	\vfill
	\begin{block}{\huge Aufgabe 3 (Umlaute)}
		Geben Sie in einem \LaTeX-Dokument die deutschen Umlaute Ä, Ö, Ü, ä, ö, ü, ß aus, indem Sie:
\begin{itemize}
	\item die Umlaute maskieren,
	\item die Umlaute im Quellcode direkt verwenden und den Zeichensatz entsprechend anpassen.
\end{itemize}
	\end{block}
\end{frame}
\begin{frame}
	\begin{block}{\huge Aufgabe 4 (Zusammensetzen des Dokuments aus mehreren Dateien)}
			Erstellen Sie ein Dokument aus den bisherigen Aufgaben, welches aus mehreren Dateien besteht. In der Hauptdatei sollen die Präambel, die Titelseite und das Inhaltsverzeichnis stehen. Jedes Kapitel der obersten Ebene soll in einer eigenen Datei stehen und in die Hauptdatei mittels \befehl{include} eingebunden werden. Hierzu müssen die Dateien der vorherigen Aufgaben in ein neues Verzeichnis kopiert werden und die Befehle \befehl{documentclass} und die \keyword{document}-Umgebung aus den vorherigen Aufgaben entfernt werden, da diese sonst doppelt vorhanden wären.
		 

	\end{block}
\end{frame}
\begin{frame}[fragile]
	\frametitle{Aufzählungen}
	\vspace{-0.5cm}
	drei Grundarten von Aufzählungen: \\
	\begin{tabular}{rl}
		\emphkeyword{itemize} & einfache Aufzählung\\
		\emphkeyword{enumerate} & nummerierte Aufzählung\\
		\emphkeyword{description} & Beschreibung
	\end{tabular}\par \vfill
	
	\begin{columns}[T]
		\begin{column}{0.31\textwidth}
			\begin{block}{\tt\bfseries itemize}
				\begin{itemize}
					\item A \item B \item C
				\end{itemize} 
				\begin{lstlisting}
\begin{itemize}
  \item A
  \item B
  \item C
\end{itemize}
				\end{lstlisting}
			\end{block}
		\end{column}
		\begin{column}{0.31\textwidth}
			\begin{block}{\tt\bfseries enumerate}
				\begin{enumerate}
					\item A \item B \item C
				\end{enumerate} 
				\begin{lstlisting}
\begin{enumerate}
  \item A
  \item B
  \item C
\end{enumerate}
				\end{lstlisting}
			\end{block}
		\end{column}
		\begin{column}{0.31\textwidth}
			\begin{block}{\tt\bfseries description}
				\begin{description}
					\item[A] \ldots \item[B] \ldots \item[C] \ldots
				\end{description} 
				\begin{lstlisting}
\begin{description}
  \item[A] \ldots
  \item[B] \ldots
  \item[C] \ldots
\end{description}
				\end{lstlisting}
			\end{block}
		\end{column}
	\end{columns}
\end{frame}

\begin{frame}[fragile]
	\frametitle{Verschachtelte Aufzählungen}
	\vspace{-0.9cm}
	\latexBeispielDirekt{Beispiel: verschachtelte Aufzählung}{examples/Verschachtelte_Aufzaehlung/Verschachtelte_Aufzaehlung}
\end{frame}
\begin{frame}
	\begin{block}{\huge Aufgabe 5 (Listen und Aufzählungen)}
		Erstellen Sie ein Kochrezept. Verwenden Sie für die Zutatenliste die Aufzählungs-Umgebung \keyword{itemize} und für die Zubereitungshinweise die \keyword{enumerate}-Umgebung.
	\end{block}
\end{frame}
\begin{frame}[fragile]
	\frametitle{Tabellen - ein einfaches Beispiel}
	\vspace{-0.9cm}
	\latexBeispielDirekt{Beispiel: einfache Tabelle}{examples/Tabelle_einfach/Tabelle_einfach}
	\vfill
	Tabellenformatierung:\\[0.2cm]
	\begin{tabular}{rlcrl}
		\emphkeyword{l} & linksbündig ausrichten &~~~~~~ & \emphkeyword{|} & einfacher vertikaler Trennstrich \\
		\emphkeyword{c} & zentriert ausrichten && \emphkeyword{||} & doppelter vertikaler Trennstrich\\
		\emphkeyword{r} & rechtsbündig ausrichten && \emphkeyword{@\{text\}} & benutzerdefiniertes Trennzeichen \\
		\emphkeyword{p\{n\}} & Spalte mit fester Breite \keyword{n}
	\end{tabular}
\end{frame}

\begin{frame}
	\frametitle{Tabellen - mehrspaltige Zellen}
	\vspace{-0.9cm}
	\latexBeispielDirekt{Beispiel: mehrspaltige Zellen}{examples/Tabelle_multicolumn/Tabelle_multicolumn}
	\vfill
	Tabellenformatierung:\\[0.2cm]
	\begin{tabular}{rl}
		\emphkeyword{\textbackslash{}hline} & horizontale Linie über die ganze Breite \\
		\emphkeyword{\textbackslash{}vline} & vertikaler Line innerhalb einer Zeile \\
		\emphkeyword{\textbackslash{}cline\{m-n\}} & horizontale Linie von Spalte \keyword{m} bis Spalte \keyword{n}\\
		\emphkeyword{\textbackslash{}multicolumn\{n\}\{format\}\{Inhalt\}} & Zelle über \keyword{n} Spalten
	\end{tabular}
\end{frame}
\begin{frame}
	\begin{block}{\huge Aufgabe 6 (Tabellen)}
		\begin{frame}[fragile]
	\frametitle{Tabellen - ein einfaches Beispiel}
	\vspace{-0.9cm}
	\latexBeispielDirekt{Beispiel: einfache Tabelle}{examples/Tabelle_einfach/Tabelle_einfach}
	\vfill
	Tabellenformatierung:\\[0.2cm]
	\begin{tabular}{rlcrl}
		\emphkeyword{l} & linksbündig ausrichten &~~~~~~ & \emphkeyword{|} & einfacher vertikaler Trennstrich \\
		\emphkeyword{c} & zentriert ausrichten && \emphkeyword{||} & doppelter vertikaler Trennstrich\\
		\emphkeyword{r} & rechtsbündig ausrichten && \emphkeyword{@\{text\}} & benutzerdefiniertes Trennzeichen \\
		\emphkeyword{p\{n\}} & Spalte mit fester Breite \keyword{n}
	\end{tabular}
\end{frame}

\begin{frame}
	\frametitle{Tabellen - mehrspaltige Zellen}
	\vspace{-0.9cm}
	\latexBeispielDirekt{Beispiel: mehrspaltige Zellen}{examples/Tabelle_multicolumn/Tabelle_multicolumn}
	\vfill
	Tabellenformatierung:\\[0.2cm]
	\begin{tabular}{rl}
		\emphkeyword{\textbackslash{}hline} & horizontale Linie über die ganze Breite \\
		\emphkeyword{\textbackslash{}vline} & vertikaler Line innerhalb einer Zeile \\
		\emphkeyword{\textbackslash{}cline\{m-n\}} & horizontale Linie von Spalte \keyword{m} bis Spalte \keyword{n}\\
		\emphkeyword{\textbackslash{}multicolumn\{n\}\{format\}\{Inhalt\}} & Zelle über \keyword{n} Spalten
	\end{tabular}
\end{frame}
	\end{block}
\end{frame}

\begin{frame}[fragile]
	\frametitle{Mathematik -- Formeln im Fließtext}
	\begin{itemize}
        \item Formeln müssen in \LaTeX{} markiert werden, damit sie korrekt interpretiert werden
		\item im Mathematik-Modus werden Leerzeichen ignoriert und Buchstabenketten als einzelne Zeichen betrachtet
	\end{itemize}
	\vfill
	\latexBeispielDirekt{Beispiel: Formeln im Fließtext}{examples/Formeln_im_Fliesstext/Formeln_im_Fliesstext}
\end{frame}

\begin{frame}[fragile]
	\frametitle{Mathematik -- Formeln in eigenem Absatz}
	\begin{itemize}
		\item soll eine Formel abgesetzt dargestellt werden, so benutzt man die \keyword{displaymath}-Umgebung oder die Kurzschreibweise \lstinline$\[...\]$
	\end{itemize}
	\vfill
	\latexBeispielDirekt{Beispiel: Formeln in eigenem Absatz}{examples/Formeln_eigener_Absatz/Formeln_eigener_Absatz}
\end{frame}

\begin{frame}[fragile]
	\frametitle{Mathematik -- nummerierte Gleichungen}
	\begin{itemize}
		\item Gleichungen können mit der \keyword{equation}-Umgebung automatisch fortlaufend nummeriert werden
		\item mittels der \keyword{align}-Umgebung können mehrzeilige Formeln nummeriert und ausgerichtet werden
		\item \befehl{nonumber} unterdrückt in der \keyword{align}-Umgebung die Nummerierung einer Zeile
	\end{itemize}
	\vfill
	\latexBeispielDirektKlein{Beispiel: mehrzeilige Gleichungen}{examples/Formeln_mehrzeilig/Formeln_mehrzeilig}
\end{frame}


\begin{frame}[fragile]
	\frametitle{Mathematik -- Unterschiede bei der Darstellung}
	\begin{itemize}
		\item Formeln werden innerhalb der \keyword{math}-Umgebung anders dargestellt als in der \keyword{displaymath}-Umgebung
	\end{itemize}
	\begin{center}
		\begin{tabular}{c|c|c}
			\LaTeX-Befehl & \keyword{math} & \keyword{displaymath} \\ \hline
			\lstinline$\lim_{x \ra \infty} \ra 0$
			&
			\begin{math}
				\lim_{x \to \infty} \ra 0
			\end{math}
			&\begin{minipage}{4cm}
				\begin{displaymath}
					\lim_{x \ra \infty} \ra 0
				\end{displaymath}
			\end{minipage}
		\end{tabular}
	\end{center}
	\begin{itemize}
		\item mit dem Befehl \befehl{displaystyle} bzw. \befehl{textstyle} kann das jeweils andere Verhalten erzwungen werden
	\end{itemize}
	\latexBeispielDirekt{Beispiel: \keyword{textstyle} und \keyword{displaystyle}}{examples/Formeln_Darstellung/Formeln_Darstellung}
\end{frame}

\begin{frame}
	\frametitle{Mathematik -- Symbole}
	
	\emphword{Logik}
	\begin{center}
		\begin{tabular}{ll|ll|ll|ll|ll}
			\befehl{exists} & $\exists$ & \befehl{forall} & $\forall$ & \befehl{neg} & $\neq$ & \befehl{in} & $\in$ & \befehl{emptyset} & $\emptyset$ \\
			\befehl{notin} & $\notin$ & \befehl{ni} & $\ni$ & \befehl{land} & $\land$ & \befehl{lor} & $\lor$ & \befehl{varnothing} & $\varnothing$\\
			\befehl{setminus} & $\setminus$ & \befehl{implies} & $\implies$ & \befehl{iff} & $\iff$ & \befehl{to} & $\to$ & \befehl{top} & $\top$
		\end{tabular}
	\end{center}\vspace{1cm}
	
	\emphword{Pfeile} \vspace{-0.9cm}
	\begin{center}
		\begin{tabular}{ll|ll|ll|ll}
			\befehl{leftarrow} & $\la$ & \befehl{rightarrow} & $\ra$ & \befehl{uparrow} & $\uparrow$ & \befehl{downarrow} & $\downarrow$ \\
			\befehl{Leftarrow} & $\La$ & \befehl{Rightarrow} & $\Ra$ & \befehl{Uparrow} & $\Uparrow$ & \befehl{Downarrow} & $\Downarrow$ \\
			\befehl{leftrightarrow} & $\leftrightarrow$ & \befehl{Leftrightarrow} & $\Leftrightarrow$ & \befehl{nearrow} & $\nearrow$ & \befehl{searrow} & $\searrow$ \\
			 \befehl{updownarrow} & $\updownarrow$ & \befehl{Updownarrow} & $\Updownarrow$ & \befehl{swarrow} & $\swarrow$ & \befehl{nwarrow} & $\nwarrow$ \\
			 \befehl{longleftarrow} & $\longleftarrow$ & \befehl{Longleftarrow} & $\Longleftarrow$ & \befehl{mapsto} & $\mapsto$ & \befehl{leadsto} & $\leadsto$ \\
			 \befehl{longrightarrow} & $\longrightarrow$ & \befehl{Longrightarrow} & $\Longrightarrow$
		\end{tabular}
	\end{center}
\end{frame}

\begin{frame}
	\frametitle{Mathematik -- Symbole}
	\emphword{Sonstige}
	\begin{center}
		\begin{tabular}{ll|ll|ll|ll|ll}
			\befehl{partial} & $\partial$ & \befehl{nabla} & $\nabla$ & \befehl{infty} & $\infty$ & \befehl{ell} & $\ell$ & \befehl{imath} & $\imath$ \\
			\befehl{jmath} & $\jmath$ & \befehl{Re} & $\Re$ & \befehl{Im} & $\Im$ & \befehl{dots} & $\dots$ & \befehl{cdots} & $\cdots$ \\
			\befehl{vdots} & $\vdots$ & \befehl{ddots} & $\ddots$ & \befehl{ldots} & $\ldots$ & \befehl{pm} & $\pm$ & \befehl{mp} & $\mp$
		\end{tabular}
	\end{center} \vspace{1cm}
	

\end{frame}


\begin{frame}
	\frametitle{Mathematik -- Griechische Buchstaben}
	\vspace{-0.5cm}
	\begin{center}
\begin{tabular}{ll|ll||ll}
\befehl{alpha}      & $\alpha$  & 
\befehl{xi}         & $\xi$ &
\befehl{Gamma}      & $\Gamma$ \\

\befehl{beta}       & $\beta$  & 
\befehl{pi}         & $\pi$ &
\befehl{Delta}      & $\Delta$ \\

\befehl{gamma}      & $\gamma$  & 
\befehl{varpi}      & $\varpi$ &
\befehl{Theta}      & $\Theta$ \\

\befehl{delta}      & $\delta$  & 
\befehl{rho}        & $\rho$ &
\befehl{Lambda}     & $\lambda$ \\

\befehl{epsilon}    & $\epsilon$  & 
\befehl{varrho}     & $\varrho$ &
\befehl{Xi}         & $\Xi$ \\

\befehl{varepsilon} & $\varepsilon$  & 
\befehl{sigma}      & $\sigma$ &
\befehl{Pi}         & $\Pi$ \\

\befehl{zeta}       & $\zeta$  & 
\befehl{varsigma}   & $\varsigma$ &
\befehl{Sigma}      & $\Sigma$ \\

\befehl{eta}        & $\eta$  & 
\befehl{tau}        & $\tau$ &
\befehl{Upsilon}    & $\Upsilon$ \\

\befehl{theta}      & $\theta$  & 
\befehl{upsilon}    & $\upsilon$ &
\befehl{Phi}        & $\Phi$ \\

\befehl{vartheta}   & $\vartheta$  & 
\befehl{phi}        & $\phi$ &
\befehl{Psi}        & $\Psi$ \\

\befehl{iota}       & $\iota$  & 
\befehl{varphi}     & $\varphi$ &
\befehl{Omega}      & $\Omega$ \\

\befehl{kappa}      & $\kappa$  & 
\befehl{chi}        & $\chi$ \\

\befehl{lambda}     & $\lambda$  & 
\befehl{psi}        & $\psi$ &
                & \\

\befehl{mu}         & $\mu$  & 
\befehl{omega}      & $\omega$ &
                & \\

\befehl{nu}         & $\nu$ &
                &       &
                & \\

\end{tabular}
\end{center}
\end{frame}

\begin{frame}
	\frametitle{Mathematik -- binäre Operatoren}
	\begin{center}
\begin{tabular}{ll|ll|ll}
\befehl{leq} & $\leq$ &
\befehl{geq} & $\geq$ &
\befehl{equiv} & $\equiv$
\\
\befehl{models} & $\models$ &
\befehl{prec} & $\prec$ &
\befehl{succ} & $\succ$
\\
\befehl{sim} & $\sim$ &
\befehl{perp} & $\perp$ &
\befehl{preceq} & $\preceq$
\\
\befehl{succeq} & $\succeq$ &
\befehl{simeq} & $\simeq$ &
\befehl{mid} & $\mid$
\\
\befehl{ll} & $\ll$ &
\befehl{gg} & $\gg$ &
\befehl{asymp} & $\asymp$ 
\\
\befehl{parallel} & $\parallel$ &
\befehl{subset} & $\subset$ &
\befehl{supset} & $\supset$
\\
\befehl{approx} & $\approx$ &
\befehl{bowtie} & $\bowtie$ &
\befehl{subseteq} & $\subseteq$
\\
\befehl{supseteq} & $\supseteq$ &
\befehl{cong} & $\cong$ &
\befehl{sqsubset} & $\sqsubset$
\\
\befehl{sqsupset} & $\sqsupset$ &
\befehl{neq} & $\neq$ &
\befehl{smile} & $\smile$
\\
\befehl{sqsubseteq} & $\sqsubseteq$ &
\befehl{sqsupseteq} & $\sqsupseteq$ &
\befehl{doteq} & $\doteq$
\\
\befehl{frown} & $\frown$ &
\befehl{in} & $\in$ &
\befehl{ni} & $\ni$
\\
\befehl{propto} & $\propto$ &
= & $=$ &
\befehl{vdash} & $\vdash$
\\
\befehl{dashv} & $\dashv$ &
< & $<$ &
> & $>$
\end{tabular}
\end{center}
\end{frame}

\begin{frame}
	\frametitle{Mathematik -- Negation von Operatoren}
	\begin{itemize}
		\item oftmals kann ein Operator negiert werden, indem der Befehl \befehl{not} vorgestellt wird
		\begin{center}
			\begin{tabular}{ll|ll|ll}
				\befehl{not<} & $\not<$ & \befehl{not}\befehl{leq} & $\not\leq$ & \befehl{not}\befehl{subseteq} $\not\subseteq$
			\end{tabular}
		\end{center}
		\item einige Operatoren haben dafür jedoch spezielle Symbole
		\begin{center}
			\begin{tabular}{ll|ll}
				\befehl{not=} & $\not=$ & \befehl{neq} & $\neq$ \\
				\befehl{not}\befehl{in} & $\not\in$ & \befehl{notin} & $\notin$
			\end{tabular}
		\end{center}
	\end{itemize}
\end{frame}

\begin{frame}
	\frametitle{Mathematik  -- gestapelte Operatoren}
	\begin{itemize}
		\item manchmal werden Anmerkungen an Relationen geschrieben oder eigene Operatoren definiert
		\item dazu eignet sich der \befehl{stackrel} Befehl
	\end{itemize}
	\vfill
	\latexBeispielDirekt{Beispiel: \befehl{stackrel} Befehl}{examples/Mathematik_stackrel/Mathematik_stackrel}
\end{frame}

\begin{frame}
	\frametitle{Mathematik -- Funktionen}
	\begin{center}
\begin{tabular}{lllll}
\befehl{arccos} \quad &
\befehl{arcsin} \quad &
\befehl{arctan} \quad &
\befehl{arg} \quad &
\befehl{cos} \\[0.5cm]
\befehl{cosh} &
\befehl{cot} &
\befehl{coth} &
\befehl{csc} &
\befehl{deg} \\[0.5cm]
\befehl{det} &
\befehl{dim} &
\befehl{exp} &
\befehl{gcd} &
\befehl{hom} \\[0.5cm]
\befehl{inf} &
\befehl{ker} &
\befehl{lg} &
\befehl{lim} &
\befehl{liminf} \\[0.5cm]
\befehl{limsup} &
\befehl{ln} &
\befehl{log} &
\befehl{max} &
\befehl{min} \\[0.5cm]
\befehl{Pr} &
\befehl{sec} &
\befehl{sin} &
\befehl{sinh} &
\befehl{sup} \\[0.5cm]
\befehl{tan} &
\befehl{tanh} &
\end{tabular}
\end{center}
\end{frame}

\begin{frame}
	\frametitle{Mathematik -- Hoch- und Tiefstellung}
	\begin{itemize}
		\item um Exponenten oder Indizes anzugeben, benutzt man zur Hochstellung das Zeichen \lstinline$^$ und zur Tiefstellung das Zeichen \lstinline$_$
		\item möchte man mehr als ein Zeichen hoch- oder tiefstellen, muss man den Term in geschweiften Klammern angeben
		\item die Reihenfolge ist egal
		\item mit den Befehlen \befehl{limits} und \befehl{nolimits} kann man die Darstellung für das Hoch- bzw. Tiefstellen beinflussen
	\end{itemize}
	\vfill
	\latexBeispielDirekt{Beispiel: Hoch- und Tiefstellung}{examples/Mathematik_HochTief/Mathematik_HochTief}
\end{frame}

\begin{frame}[fragile]
	\frametitle{Mathematik -- Brüche}
	\begin{itemize}
		\item der \befehl{frac} Befehl dient der Darstellung von Brüchen und erfordert zwei Argumente
	\end{itemize}
	\vfill
	\latexBeispielDirekt{Beispiel: Brüche}{examples/Mathematik_frac/Mathematik_frac}
\end{frame}

\begin{frame}[fragile]
	\frametitle{Mathematik -- Wurzel und Binomialkoeffizient}
	\emphword{Wurzel}
	\begin{itemize}
		\item der Wurzelbefehl heißt \befehl{sqrt}
		\item optional kann der Parameter \keyword{n} angeben werden ($n = 2$:  Quadratwurzel)
	\end{itemize}
	\vfill
	\emphword{Binomialkoeffizient}
	\begin{itemize}
		\item zur Darstellung des Binomialkoeffizienten dient der Befehl \befehl{choose}
		\item das erste Argument wird oben, das zweite unten gesetzt
	\end{itemize}
	\vfill
	\latexBeispielDirekt{Beispiel: Wurzeln, Binomialkoeffizienten}{examples/Mathematik_Wurzel_Binom/Mathematik_Wurzel_Binom}
\end{frame}

\begin{frame}[fragile]
	\frametitle{Mathematik -- Klammerung} \vspace{-0.6cm}
	\begin{itemize}
		\item Klammern: \keyword{() [] \{\} ||} \befehl{langle}\befehl{rangle} \befehl{lfloor}\befehl{rfloor} \befehl{lceil}\befehl{rceil}
			\[ (\,) \quad [\,] \quad \{\,\} \quad |\,| \quad \langle \, \rangle \quad \lfloor \, \rfloor \quad \, \lceil \, \rceil \]
		\item die Größe der Klammern kann manuell durch die Befehle \befehl{big}, \befehl{Big}, \befehl{bigg}, \befehl{Bigg} variert werden
			\[ ( \big( \Big( \bigg( \Bigg( \, \Bigg) \bigg) \Big) \big) ) \]
		\item besser: die Klammergröße kann automatisch gesetzt werden
		\item dazu setzt man den einzuklammernden Begriff zwischen die Befehle \befehl{left} und \befehl{right}; direkt darauf folgt das zu verwendende Klammerzeichen\\
		\item möchte man auf einer Seite keine Klammern haben, verwendet man als Klammerzeichen \keyword{.} (Punkt)
		\item Beispiel: \lstinline$\left\{ 1 - \left| \frac{1}{2} \right| \dots \right.$
		\[
			\left\{ 1 - \left| \frac{1}{2} \right| \dots \right.
		\]
	\end{itemize}
\end{frame}

\begin{frame}[fragile]
	\frametitle{Mathematik -- Matrizen}
	\begin{itemize}
		\item die \keyword{array}-Umgebung entspricht der \keyword{tabular}-Umgebung im Mathematik-Modus
		\item die Inhalte der Zellen sind automatisch im Mathematik-Modus gesetzt
		\item bekannte Befehle für Tabellen (z. B. \befehl{multicolumn}, \befehl{hline}, \dots) können verwendet werden
	\end{itemize}
	\vfill
	\latexBeispielDirekt{Beispiel: Matrizen}{examples/Mathematik_Matrix/Mathematik_Matrix}
\end{frame}

\begin{frame}[fragile]
	\frametitle{Mathematik -- Fallunterscheidungen}
	\begin{itemize}
		\item für Fallunterscheidungen steht die \keyword{cases}-Umgebung zur Verfügung
		\item kann auch mittels Klammerung und der \keyword{array}-Umgebung selbst erstellt bzw. individualisiert werden
	\end{itemize}
	\vfill
	\latexBeispielDirekt{Beispiel: Fallunterscheidungen}{examples/Mathematik_cases/Mathematik_cases}
\end{frame}

\begin{frame}[fragile]
	\frametitle{Mathematik -- Schriftgröße}
	\begin{itemize}
		\item die Schriftgröße kann im Mathematik-Modus durch folgende Befehl geändert werden:
		\begin{center}
			\begin{tabular}{ll}
				\befehl{displaystyle} & $\displaystyle \frac{1}{2}$ \\[0.5cm]
				\befehl{textstyle} & $\textstyle \frac{1}{2}$ \\[0.5cm]
				\befehl{scriptstyle} & $\scriptstyle \frac{1}{2}$
			\end{tabular}
		\end{center}
	\end{itemize}
\end{frame}

\begin{frame}[fragile]
	\frametitle{Mathematik -- Akzente}
	\begin{center}
		\begin{tabular}{ll|ll|ll}
			\verb|a'| & $a'$ &
			\verb|a''| & $a''$ &
			\verb|a'''| & $a'''$ \\
			\verb|\bar{a}| & $\bar{a}$ &
			\verb|\overline{a}| & $\overline{a}$ &
			\verb|\underline{a}| & $\underline{a}$ \\
			\verb|\hat{a}| & $\hat{a}$ &
			\verb|\widehat{ab}| & $\widehat{ab}$ &
			\verb|\check{a}| & $\check{a}$ \\
			\verb|\tilde{a}| & $\tilde{a}$ &
			\verb|\widetilde{ab}| & $\widetilde{ab}$ &
			\verb|\vec{a}| & $\vec{a}$ \\
			\verb|\dot{a}| & $\dot{a}$ &
			\verb|\ddot{a}| & $\ddot{a}$
		\end{tabular}
	\end{center}
\end{frame}

\begin{frame}[fragile]
	\frametitle{Mathematik -- horizontale Abstände}
	\begin{center}
		\begin{tabular}{l|c|l}
		Befehl & Breite & Beschreibung\\ \hline
			\verb|\qquad| & $\overline{\qquad}$ & $2\times$ quad \\
			\verb|\quad| & $\overline{\quad}$ & so breit wie ein Zeichen hoch ist \\
			\textvisiblespace~ (Leerzeichen)          & $\overline{~}$ & Zeichenabstand \\
			\verb|\,| & $\overline{\,}$ & $\frac{3}{18} \times$ quad \\
			\verb|\:| & $\overline{\:}$ & $\frac{4}{18} \times$ quad \\
			\verb|\;| & $\overline{\;}$ & $\frac{4}{18} \times$ quad \\
			\verb|\!| & $\overline{\,}$ & $-\frac{3}{18} \times$ quad \\
		\end{tabular}
	\end{center}
	\vfill
	\latexBeispielDirekt{Beispiel: horizontaler Abstand}{examples/Mathematik_H_Abstand/Mathematik_H_Abstand}
\end{frame}

\begin{frame}
	\begin{block}{\huge Aufgabe 7 (Formelsatz)}
		\large
		Erstellen Sie folgenden Text:
	\begin{center}
		\shadowbox{
			\begin{minipage}{0.9\textwidth}
				Sei $I$ ein reelles Intervall und $f:I\to\mathbb{R}$ eine $(n+1)$-mal stetig differenzierbare Funktion. Dann gilt für alle $a, x\in I$:
				\(f(x) = T_n(x) + R_n(x)\)
				mit dem $n$-ten Taylorpolynom an der Entwicklungsstelle $a$
				\begin{eqnarray*}
				T_(x) & = & \sum_{k=0}^n \frac{f^{(k)}(x)}{k!}(x-a)^k \\
				      & = & f(a) + \frac{f'(a)}{1!} (x-a) + \frac{f''(a)}{2!} (x-a)^2 + \cdots + \frac{f^{(n)}(x)}{n!} (x-a)^n
				\end{eqnarray*}
				und dem $n$-ten Restglied
				\[R_n(x) = \int_a^x \frac{(x-t)^n}{n!} f^{(n+1)}(t)\;\mathrm{d}t\]
				
				In den Formeln stehen $f'$, $f''$, \dots, $f^{(n)}$ für die erste, zweite, ..., $n$-te Ableitung der Funktion $f$.
			\end{minipage}
		}
	\end{center}
	\textit{Hinweis:} Verwenden Sie zur Erzeugung der Mengensymbole $\mathbb{R}$ und $\mathbb{N}$ den im Paket \keyword{amssymb} {enthaltenen} Befehl \befehl{mathbb} und für die ausgerichteten Gleichungen die Umgebung \keyword{align} aus dem \keyword{amsmath} Paket.
	\end{block}
\end{frame}
\begin{frame}[fragile]
	\frametitle{Grafiken}
	\begin{itemize}
		\item Grafiken können mit dem Befehl \befehl{includegraphics} eingebunden werden
		\item dieser benötigt das Paket \keyword{graphicx}, dass in der Präambel geladen werden muss:\\
		\lstinline$\usepackage{graphicx}$
		\item benutzt man \keyword{pdflatex} zum Kompilieren des Dokuments, so müssen die verwendeten Grafiken in PDF- oder Bitmap-Grafiken umgewandelt werden
		\item alternativ kann man auch das Paket \keyword{epstopdf} verwendet, dann muss \keyword{pdflatex} aber so gestartet werden:\\
		\kommandozeile{pdflatex --shell-escape}
	\end{itemize}
\end{frame}

\begin{frame}[fragile]
	\frametitle{Grafiken -- Beispiel} \vspace{-0.5cm}
	\latexBeispielDirekt{Beispiel: \befehl{includegraphics}}{examples/Grafiken_Beispiel/Grafiken_Beispiel} \par \vspace{1cm}
	\emphword{Optionale Parameter}
	\begin{center}
		\begin{tabular}{ll}
			\keyword{width=b} & Breite vorgeben  (z. B. in \keyword{cm})\\
			\keyword{height=h} & Höhe vorgeben (z. B. in \keyword{cm})\\
			\keyword{keepaspectratio=k} & Seitenverhältnis beibehalten (\keyword{true} oder \keyword{false}) \\
			\keyword{scale=s} & skalieren  (Skalierungsfaktor angeben)\\
			\keyword{angle=a} & rotieren (Winkel in Grad angeben) \\
			\keyword{trim=l b r t} & Bild zuschneiden
		\end{tabular}
	\end{center}
\end{frame}

\begin{frame}[fragile]
	\frametitle{Grafiken -- \keyword{picture}-Umgebung}
	\latexBeispielDirektKlein{Beispiel: \keyword{picture}-Umgebung}{examples/Grafiken_pictures/Grafiken_pictures}
\end{frame}

\begin{frame}[fragile]
	\frametitle{Gleitende Objekte (Floats)}
	\begin{itemize}
		\item gleitende Objekte müssen nicht an der Stelle im Text auftauchen, an der sie definiert wurden
        \item \LaTeX{} entscheidet ``selbstständig'' wo sie plaziert werden
		\item Beispielumgebungen (die gleitende Objekte darstellen):
		\begin{itemize}
			\item \befehl{begin\{figure\}[htbp] ...}\befehl{end\{figure\}}
			\item \befehl{begin\{table\}[htbp] ...}\befehl{end\{table\}}
		\end{itemize}
		\item optional kann die gewünschte Platzierung angegeben werden:
		\begin{center}
			\begin{tabular}{lll}
				\emphkeyword{h} & \textit{here} & Objekt dort einbauen, wo es definiert wird\\
				\emphkeyword{t} & \textit{top} & Objekt am Anfang der Seite platzieren \\
				\emphkeyword{b} & \textit{bottom} & objekt am Ende der Seite platzieren\\
				\emphkeyword{p} & \textit{page} & Gleitobjekte auf einer Seite sammeln
			\end{tabular}
		\end{center}
		\item zur Beschriftung dient der \befehl{caption}-Befehl
	\end{itemize}
\end{frame}


\begin{frame}[fragile]
  \frametitle{Inhaltsverzeichnis ausgeben}
  \begin{itemize}
    \item um Standardverzeichnisse auszugeben, muss man einfach die entsprechenden Schlüsselwörter angeben:
      \begin{itemize}
        \item \befehl{tableofcontents}
        \item \befehl{listoffigures}
        \item \befehl{listoftables}
      \end{itemize}
    \item die maximale Tiefe des Inhaltsverzeichnisses kann mit folgendem Befehl geändert werden:
      \begin{itemize}
        \item \befehl{setcounter\{tocdepth\}\{1\}}
      \end{itemize}
    \item manuelle Einträge können so eingefügt werden:
      \begin{itemize}
        \item \befehl{addcontentsline\{toc\}\{subsection\}\{Titel\}}
      \end{itemize}
    \item mögliche Werte für das Zielverzeichnis:
      \begin{center}
        \begin{tabular}{lll}
          \emphkeyword{toc} & \textit{table of contents} &\keyword{chapter}, \keyword{section}, \keyword{subsection} \\ && \keyword{subsubsection}, \keyword{paragraph} \\
          \emphkeyword{lof} & \textit{list of figures} & \keyword{figure} \\
          \emphkeyword{lot} & \textit{list of tables} & \keyword{table}
        \end{tabular}
      \end{center}
  \end{itemize}
\end{frame}


\begin{frame}[fragile]
	\frametitle{Querverweise}
	\begin{itemize}
		\item Befehle zur Benutzung von Querverweisen:
		\begin{center}
			\begin{tabular}{ll}
				\befehl{label\{marker\}} & Verbindet die momentane Textstelle mit \keyword{marker}. So kann\\
				& von einer anderen Textstelle aus auf diese Stelle verwiesen\\
				& werden.\\[0.5cm]
				\befehl{ref\{marker\}} & Erzeugt einen Querverweis auf eine Stelle, die zuvor mittels\\
				& \keyword{label} gekennzeichnet wurde. Der Querverweis gibt die\\
				& Gliederungsnummer der betreffenden Textstelle an.\\[0.5cm]
				\befehl{pageref\{marker\}} & Wie \befehl{ref}, gibt jedoch die Seitennummer der Textstelle\\
				& zurück.
			\end{tabular}
		\end{center}
	\end{itemize}
\end{frame}

\begin{frame}[fragile]
	\frametitle{Querverweise}
	\begin{itemize}
		\item Durch ein Präfix (optional) kann angegeben werden, was referenziert wird, z. B.:\\
		\befehl{label\{sec:Querverweise\}}
		\item mögliche Präfixe sind:
		\begin{center}
			\begin{tabular}{rl}
				\emphkeyword{chap} & chapter\\
				\emphkeyword{sec} &  section\\
				\emphkeyword{fig} & figure\\
				\emphkeyword{tab} & table\\
				\emphkeyword{eq} & equation\\
				\emphkeyword{lst} & listing
			\end{tabular}
		\end{center}
		\item in \keyword{figure}- und \keyword{table}-Umgebung muss das Label innerhalb des \befehl{caption}-Befehls gesetzt werden
	\end{itemize}
\end{frame}

\begin{frame}[fragile]
	\frametitle{Querverweise -- Beispiel}
	
	\latexBeispielDirekt{Querverweise -- Beispiel}{examples/Querverweise/Querverweise.tex}
\end{frame}

\begin{frame}
	\begin{block}{\huge Aufgabe 8 (Inhaltsverzeichnis und Querverweise)}
		Ändern Sie das Dokument aus Aufgabe 2 so ab, dass eine Titelseite und ein Inhaltsverzeichnis ausgegeben wird. Bauen Sie einige Querverweise mit den Befehlen \befehl{label}, \befehl{ref} und \befehl{pageref} in das Dokument ein.
	\end{block}
	\begin{block}{\huge Aufgabe 9 (Bilder)}
		Binden Sie ein Bild in ein das \LaTeX-Dokument aus Aufgaben 4 ein. Es soll eine Breite von \keyword{6 cm} haben und in einer Figure-Umgebung mit Bezeichnung stehen. Verändern Sie die Positionierung der gleitenden Umgebung \keyword{figure}. Erstellen Sie einen Querverweis auf das Bild.
	\end{block}
\end{frame}
\begin{frame}[fragile]
	\frametitle{Eigene \LaTeX-Befehle}
	\begin{itemize}
		\item eigene Befehle können mittels \befehl{newcommand} angelegt werden
		\item allgemeine Form:
		\begin{itemize}
			\item \befehl{newcommand\{befehlsname\}\{definition\}}
			\item \befehl{newcommand\{befehlsname\}[n]\{definition\}}
			\item \befehl{newcommand\{befehlsname\}[n][default]\{definition\}}
		\end{itemize}
		\begin{center}
			\begin{tabular}{rl}
				\emphkeyword{befehlsname} & Name des Befehls (muss mit \textbackslash{} beginnen)\\
				\emphkeyword{n} & Anzahl der Parameter \\
				\emphkeyword{default} & Vorgabewert für optionale Parameter\\
				\emphkeyword{definition} & alles was beim Aufruf ausgeführt werden soll
			\end{tabular}
		\end{center}
		\item mit \befehl{renewcommand} kann ein bereits vorhandener Befehl ersetzt werden
	\end{itemize}
\end{frame}

\begin{frame}[fragile]
	\frametitle{Eigene \LaTeX-Befehle -- Beispiel}
	
	\latexBeispielDirekt{Eigene \LaTeX-Befehle -- Beispiel}{examples/Eigene_Befehle/Eigene_Befehle}
\end{frame}

\begin{frame}[fragile]
	\frametitle{Eigene \LaTeX-Umgebungen}
	\begin{itemize}
		\item eigene Umgebungen können mittels \befehl{newenvironment} angelegt werden
		\item allgemeine Form:
		\begin{itemize}
			\item \befehl{newenvironment\{umgebung\}\{vorher\}\{nachher\}}
			\item \befehl{newenvironment\{umgebung\}[n]\{vorher\}\{nachher\}}
			\item \befehl{newenvironment\{umgebung\}[n][default]\{vorher\}\{nachher\}}
		\end{itemize}
		\begin{center}
			\begin{tabular}{rl}
				\emphkeyword{umgebung} & Name der Umgebung (ohne \textbackslash{}) \\
				\emphkeyword{n} & Anzahl der Parameter \\
				\emphkeyword{vorage} & Vorgabewert für optionale Parameter\\
				\emphkeyword{vorher} & Befehle, die vor Beginn der Umgebung ausgeführt werden\\
				\emphkeyword{nachher} & Befehle, die nach Ende der Umgebung ausgeführt werden
			\end{tabular}
		\end{center}
		\item mit \befehl{renewenvironment} kann eine bereits vorhandene Umgebung ersetzt werden
	\end{itemize}
\end{frame}

\begin{frame}[fragile]
	\frametitle{Eigene \LaTeX-Umgebungen -- Beispiel}
	
	\latexBeispielDirekt{Eigene \LaTeX-Umgebungen -- Beispiel}{examples/Eigene_Umgebung/Eigene_Umgebung}
\end{frame}

\begin{frame}[fragile]
	\frametitle{Eigene Zähler}
\vspace{-1cm}
	\begin{center}
		\begin{tabular}{lll}
			\befehl{newcounter\{zaehler\}} & \multicolumn{2}{l}{neuen Zähler namens \keyword{zaehler} anlegen}\\
			\befehl{newcounter\{zaehler\}[depend]} & \multicolumn{2}{l}{neuen Zähler namens \keyword{zaehler} anlegen, der bei}\\& \multicolumn{2}{l}{Veränderung von \keyword{depend} zurückgesetzt wird}\\
			\befehl{stepcounter\{zaehler\}} & \multicolumn{2}{l}{Zähler \keyword{zaehler} um eines erhöhen}\\
			\befehl{setcounter\{zaehler\}\{wert\}} & \multicolumn{2}{l}{Zähler \keyword{zaehler} auf Wert \keyword{wert} setzen}\\
			\befehl{addtocounter\{zaehler\}\{wert\}} & \multicolumn{2}{l}{Zähler \keyword{zaehler} um \keyword{wert} erhöhen} \\
			\befehl{value\{zaehler\}} & \multicolumn{2}{l}{den Wert von \keyword{zaehler} auslesen}\\
			\befehl{arabic\{zaehler\}} & Ziffer & 1, 2, 3, 4, \dots \\
			\befehl{roman\{zaehler\}} & Römische Zahl (klein) & i, ii, iii, iv, \dots \\
			\befehl{Roman\{zaehler\}} & Römische Zahl (groß) \hspace{1cm} & I, II, III, IV, \dots \\
			\befehl{alph\{zaehler\}} & Kleinbuchstaben & a, b, c, d, \dots \\
			\befehl{Alph\{zaehler\}} & Großbuchstaben & A, B, C, D, \dots
		\end{tabular}
	\end{center}
\end{frame}


\begin{frame}[fragile]
	\frametitle{Vordefinierte Zähler}
	
	\begin{center}
		\begin{tabular}{ll}
			\emphword{Dokument} & \keyword{page} \\[0.5cm]
			\emphword{Kapitel} & \keyword{part}, \keyword{chapter}, \keyword{section}, \keyword{subsection}, \keyword{subsubsection} \\
			& \keyword{paragraph}, \keyword{subparagraph} \\[0.5cm]
			\emphword{Aufzählung} & \keyword{enumi}, \keyword{enumii}, \keyword{enumiii}, \keyword{enumiv} \\[0.5cm]
			\emphword{Umgebungen} & \keyword{equation}, \keyword{figure}, \keyword{table}, \keyword{footnote}
		\end{tabular}
	\end{center}
\end{frame}

\begin{frame}[fragile]
	\frametitle{Eigene Längen}
	
	\begin{center}
		\begin{tabular}{ll}
			\befehl{newlength\{}\befehl{laenge\}} & neue Längenvariable anlegen \\
			\befehl{setlength\{}\befehl{laenge\}\{x\}} & Länge auf \keyword{x} setzen \\
			\befehl{addtolength\{}\befehl{laenge\}\{x\}} & Länge um \keyword{x} erhöhen\\
			\befehl{settolength\{}\befehl{laenge\}\{text\}} & Länge von \keyword{text} speichern\\
			\befehl{settoheight\{}\befehl{laenge\}\{text\}} & Höhe von \keyword{text} speichern \\
			\befehl{settodepth\{}\befehl{laenge\}\{text\}} & Größter Abstand zur Basislinie
		\end{tabular}
	\end{center}
\end{frame}

\begin{frame}[fragile]
	\frametitle{Vordefinierte Längen}
		\begin{columns}[T]
		\begin{column}{0.54\textwidth} \vspace{-1cm}
			\begin{enumerate}
				\item \befehl{hoffset}
				\item \befehl{voffset}
				\item \befehl{oddsidemargin}
				\item \befehl{topmargin}
				\item \befehl{headheight}
				\item \befehl{headsep}
				\item \befehl{textheight}
				\item \befehl{textwidth}
				\item \befehl{marginparsep}
				\item \befehl{marginparwidth}
				\item \befehl{marginpar}
				\item \befehl{footskip}
				\item[~] \befehl{paperwidth}, \befehl{paperheight}
			\end{enumerate}
		\end{column}
		\begin{column}{0.44\textwidth} \vspace{-3cm}
			\includegraphics[scale=1.2]{img/page_layout.pdf}
		\end{column}
	\end{columns}
\end{frame}

\begin{frame}[fragile]
	\frametitle{Theorem-Umgebung}
	\begin{itemize}
		\item der \befehl{newtheorem} Befehl dient der Erzeugung von Umgebungen für Theoreme, Sätze, Definitionen etc.
		\begin{itemize}
			\item \befehl{newtheorem\{name\}\{beschriftung\}}
			\item \befehl{newtheorem\{name\}\{beschriftung\}[zaehler]}
		\end{itemize}
		\begin{center}
			\begin{tabular}{rl}
				\emphkeyword{name} & Name der Theorem-Umgebung\\
				\emphkeyword{beschriftung} & die Bezeichnung der Umgebung im Dokument\\
				\emphkeyword{zaehler} & der Zähler der zur Nummerierung verwendet wird
			\end{tabular}
		\end{center}
	\end{itemize}
\end{frame}

\begin{frame}
	\frametitle{Theorem-Umgebung -- Beispiel}
	
	\latexBeispielDirekt{Theorem-Umgebung -- Beispiel}{examples/Theorem_Umgebung/Theorem_Umgebung}
\end{frame}
\begin{frame}[fragile]
	\frametitle{Bibliographie}
	\begin{itemize}
        \item \LaTeX{} stellt mit BibTeX ein sehr mächtiges System zur Verwaltung von Literaturverweisen bereit.
		\item Um BibTeX nutzen zu können, muss zunächst eine Datenbank angelegt werden.
		\item Die Datenbank ist eine einfache Textdatei mit Einträgen für die verschiedenen zitierten Quellen.
		\item allgemeiner Aufbau eines Eintrags:
		\begin{itemize}
			\item \keyword{@literaturtyp\{kennung, name1=''Wert1'', name2=''Wert2'', ...\}}
		\end{itemize}
		\begin{center}
			\begin{tabular}{rl}
				\emphkeyword{literaturtyp} & spezifiziert die Art der Quelle\\
				& unterschiedliche Eintragstypen erfordern unterschiedliche\\
				& Angaben zur Quelle \\[0.2cm]
				\emphkeyword{kennung} & damit kann auf den Eintrag referenziert werden\\[0.2cm]
				\emphkeyword{name=''wert''} & Zuweisung von Werten an die verschiedenen Felder
			\end{tabular}
		\end{center}
	\end{itemize}
\end{frame}

\begin{frame}[fragile]
	\frametitle{Bibliographie -- Unterstützte Literatur-Typen}
	\begin{center}
		\begin{tabular}{rl}
			\emphkeyword{article} & Veröffentlichung in einer Zeitschrift\\
			\emphkeyword{book} & Buch\\
			\emphkeyword{booklet} & Buch ohne Verleger\\
			\emphkeyword{inbook} & Teil eines Buches\\
			\emphkeyword{incollection} & Teil einer Buchreihe\\
			\emphkeyword{inproceedings} & Teil einer Veröffentlichung zu einer Konferenz\\
			\emphkeyword{manual} & Technische Dokumentation \\
			\emphkeyword{masterthesis} & Masterarbeit\\
			\emphkeyword{phdthesis} & Doktorarbeit\\
			\emphkeyword{proceedings} & Veröffentlichung zu einer Konferenz\\
			\emphkeyword{techreport} & Technischer Bericht\\
			\emphkeyword{unpublished} & unveröffentlicht\\
			\emphkeyword{misc} & falls alles andere nicht passt
		\end{tabular}
	\end{center}
\end{frame}	

\begin{frame}[fragile]
	\frametitle{Bibliographie -- Unterstützte Felder}
	\vspace{-0.7cm}
	\begin{center}
		\begin{tabular}{rl}
			\emphkeyword{address} & die Adresse des Verlags oder einer anderen Institution\\
			\emphkeyword{annotate} & Anmerkungen \\
			\emphkeyword{author} & Namen der Autoren (in BibTeX Format)\\
			& \tabitem mehrere Namen werden durch \keyword{AND} getrennt \\
			& \tabitem zwei Möglichkeiten Namen zu schreiben:\\
			& ~~~ \keyword{Donald E. Knuth} \textit{oder} \keyword{Knuth, Donald E.}\\
			\emphkeyword{booktitle} & Titel des Buchs \\
			\emphkeyword{chapter} & Kapitel- oder Abschnitt-Nummer\\
			\emphkeyword{crossref} & Datenbank-Schlüssel\\
			\emphkeyword{edition} & Auflage (z. B. eines Buchs)\\
			\emphkeyword{editor} & Namen der Editoren (analog dem \keyword{author}-Feld)\\
			\emphkeyword{howpublished} & Veröffentlichungsart \\
			\emphkeyword{institution} & fördernde Institution eines technischen Reports\\	
			\emphkeyword{journal} & Zeitschriftenname (häufig abgekürzt)
		\end{tabular}
	\end{center}
\end{frame}

\begin{frame}[fragile]
	\frametitle{Bibliographie -- Unterstützte Felder}
	\vspace{-0.7cm}
	\begin{center}
		\begin{tabular}{rl}
            \emphkeyword{key} & Feld zur Sortierung und Erstellung von Labels\\
			\emphkeyword{month} & Monat der Veröffentlichung/Erscheinung\\
			\emphkeyword{note} & zusätzliche Information\\
			\emphkeyword{number} & Nummer einer Zeitschrift, eines Reports oder eines Bandes\\
			\emphkeyword{organization} & fördernde Organisation\\
			\emphkeyword{pages} & Seitenzahlen oder Seitenzahlbereich (z. B. \keyword{7-33})\\
			\emphkeyword{publisher} & Name des Verlags\\
			\emphkeyword{school} & Name der ``Schule'', an der eine Abschlussarbeit geschrieben wurde\\
			\emphkeyword{series} & Name einer Reihe\\
			\emphkeyword{title} & Titel des Werkes\\
			\emphkeyword{type} & Typ eines technischen Reports\\
			\emphkeyword{volume} & Band einer Zeitschrift oder eines Buches\\
			\emphkeyword{year} & Jahr der Veröffentlichung/Erscheinung
		\end{tabular}
	\end{center}
\end{frame}

\begin{frame}[fragile]
	\frametitle{Bibliographie einbinden}
	\begin{itemize}
		\item verschiedene Felder sind bei verschiedenen Literaturtypen vorgeschrieben
		\item es gibt noch weitere Felder wie \keyword{ISBN}, \keyword{doi}, \keyword{abstract}, ...
		\item Sonderzeichen und Umlaute müssen speziell maskiert werden\\[0.5cm]
		\item BibTeX muss eigens aufgerufen werden:\\
		\kommandozeile{latex beispiel}\\
		\kommandozeile{bibtex beispiel}\\
		\kommandozeile{latex beispiel}\\
		\kommandozeile{latex beispiel}
		\item BibTeX prüft die Syntax der Einträge und erzeugt eine Datei, die in das \LaTeX-Dokument eingebunden werden kann
        \item nach Aufruf von BibTeX muss \LaTeX{} die Datei zwei mal kompilieren, damit alle Referenzen korrekt gesetzt werden
	\end{itemize}
\end{frame}

\begin{frame}[fragile]
	\frametitle{Bibliographie einbinden}
	\begin{itemize}
		\item um die Bibliographie anzuzeigen, verwendet man den folgenden Befehl:\\
		\befehl{bibliography\{literatur\}}
		\item \keyword{literatur} bezeichnet dabei die Datei mit den entsprechenden Literaturangaben
		\item aufgelistet werden alle Quellen, die im Dokument zitiert wurden (mittels dem Befehl \befehl{cite\{kennung\}})
		\item durch \befehl{nocite\{kennung\}} wird auch der \keyword{kennung} entsprechende Eintrag aufgeführt, auch wenn er nicht zitiert wurde
		\item die Formatierung hängt von \befehl{bibliographystyle} ab
		\item es gibt verschiedene Pakete zur Gestaltung der Literaturverweise, z. B. \keyword{natbib}
	\end{itemize}
\end{frame}

\begin{frame}[fragile]
	\frametitle{Bibliographie mit \emphkeyword{natbib}}
	\vspace{-1cm}
	\begin{itemize}
		\item \keyword{natbib} wird durch \befehl{usepackage\{natbib\}} (in der Präambel) eingebunden
		\item als Format wird dann \keyword{plainnat} gewählt:
		\befehl{bibliographystyle\{plainnat\}}
		\item \keyword{natbib} unterstützt mehrere zusätzliche Literaturangaben wie
		\begin{center}
			\begin{tabular}{rl}
				\emphkeyword{ISBN} & ISBN-Nummer eines Buches\\
				\emphkeyword{ISSN} & ISSN-Nummer einer Zeitschrift\\
				\emphkeyword{URL} & Internet-Adresse für Online-Dokumente\\
				\emphkeyword{DOI} & Digital Object Identifier
			\end{tabular}
		\end{center}
	\end{itemize}
	
	\begin{columns}[c]
		\begin{column}{0.54\textwidth}
			\begin{block}{Zitieren mit \keyword{natbib}}
			\lstinputlisting{examples/NatBib/NatBib}
			\end{block}
		\end{column}
		\begin{column}{0.44\textwidth}
			\shadowbox{\begin{minipage}{0.99\textwidth} Knuth et al. (1993)\\(Knuth et al., 1993)\\(siehe Knuth et al., 1993, S. 13)\\Knuth et al.\\1993 \end{minipage}}
		\end{column}
	\end{columns}
\end{frame}

\begin{frame}[fragile]
	\frametitle{\LaTeX\ Beamer}
	\begin{itemize}
		\item Klasse zur Erstellung von Präsentationen mit \LaTeX
		\item Benutzerhandbuch zur Klasse:
		\href{http://ftp.fau.de/ctan/macros/latex/contrib/beamer/doc/beameruserguide.pdf}{http://ftp.fau.de/ctan/macros/latex/contrib/beamer/doc/beameruserguide.pdf} \\[0.5cm]
		\item Grundkonzept: eine \emphkeyword{frame} entspricht einer Seite (Folie)
		\item innerhalb der Frames können (fast alle) normalen \LaTeX-Befehle verwendet werden
	\end{itemize}
\end{frame}

\begin{frame}[fragile]
	\frametitle{\LaTeX\ Beamer -- Präambel}
	\begin{block}{\LaTeX\ Beamer -- Beispiel - Präambel}
		\begin{lstlisting}
\documentclass{beamer}
\usepackage[ngerman]{babel}
\usepackage[utf8]{inputenc}

\usetheme{Luebeck}
\usecolortheme{orchid}
\usefonttheme{default}
\useinnertheme{rounded}
\useoutertheme{shadow}
		\end{lstlisting}
	\end{block}
\end{frame}


\begin{frame}
	\frametitle{\LaTeX\ Beamer -- Hello World}
	\latexBeispielDatei{Hello World Frame}{examples/Beamer_HelloWorld/beamer_helloworld}
\end{frame}

\begin{frame}[fragile]
	\frametitle{\LaTeX\ Beamer -- Frames}
	
	\begin{block}{\LaTeX Beamer -- Frames -- Allgemeine Form}
		\begin{lstlisting}
\begin{frame}[Overlay][Optionen]{Titel}{Untertitel}
  Inhalt
\end {frame}
		\end{lstlisting}
	\end{block}
	\begin{center}
		\begin{tabular}{rl}
			\emphkeyword{Overlay} & \keyword{<+->} sorgt dafür, dass Listen schrittweise aufgebaut werden\\
			\emphkeyword{Optionen} & Anzeigeoptionen für diese Folie (mehrere Optionen müssen\\
			& durch Kommas abgetrennt werden)\\
			\emphkeyword{Titel} & Titel der Folie; kann auch mit \befehl{frametitle} gesetzt werden\\
			\emphkeyword{Untertitel} & Untertitel der Folie; kann auch mit \befehl{framesubtitle}\\
			& gesetzt werden
		\end{tabular}
	\end{center}
\end{frame}

\begin{frame}[fragile]
	\frametitle{\LaTeX\ Beamer -- Frames -- Optionen}
	\begin{center}
		\begin{tabular}{rl}
			\emphkeyword{t} & Ausrichtung des Inhalts oben (\textit{top})\\
			\emphkeyword{c} & Ausrichtung des Inhalts mittig (\textit{center})\\
			\emphkeyword{b} & Ausrichtung des Inhalts unten (\textit{bottom})\\
			\emphkeyword{label=name} & Label für die Folie setzen\\
			\emphkeyword{plain} & Kopf- und Fußzeile unterdrücken\\
			\emphkeyword{squeeze} & Inhalt zusammenrücken\\
			\emphkeyword{fragile} & notwendig für Folien mit Quelltext (\keyword{verbatim})
		\end{tabular}
	\end{center}
\end{frame}

\begin{frame}[fragile]
	\frametitle{\LaTeX\ Beamer -- Blöcke}
	\begin{block}{Block}
		erzeugt durch:
		\begin{lstlisting}
			\begin{block}{Block}
			...
			\end{block}
		\end{lstlisting}
	\end{block}
	\begin{exampleblock}{Beispiel}
		erzeugt durch \keyword{exampleblock}-Umgebung
	\end{exampleblock}
	\begin{alertblock}{Wichtig}
		erzeugt durch \keyword{alertblock}-Umgebung
	\end{alertblock}
	\vfill
	\begin{itemize}
		\item nützlich um thematisch zusammenzufassen
		\item das Aussehen variiert je nach Themenvorlage und eigenen Einstellungen
	\end{itemize}
\end{frame}

\begin{frame}[fragile]
	\frametitle{\LaTeX\ Beamer -- Spalten}
	\vspace{-1cm}
	\begin{columns}
		\column{0.5\textwidth}{
			\begin{block}{Spalte 1}
				...
			\end{block}}
		\column{0.5\textwidth}{\begin{block}{Spalte 2}
				...
			\end{block}}
	\end{columns}
	\vfill
	\begin{block}{Quellcode}
		\begin{lstlisting}
\begin{columns}
  \column{0.5\textwidth}{
    \begin{block}{Spalte 1}
      ...
    \end{block}}
  \column{0.5\textwidth}{
    \begin{block}{Spalte 2}
      ...
    \end{block}}
\end{columns}
		\end{lstlisting}
	\end{block}
	\begin{itemize}
		\item auch mehr als zwei Spalten möglich
	\end{itemize}
\end{frame}

\begin{frame}[fragile]
	\frametitle{\LaTeX\ Beamer -- Overlays}
	\begin{itemize}
		\item Frames können mehrere Overlays enthalten.
		\item Overlays sorgen dann für das stückweise "Aufbauen" einer Folie.
		\item Der \LaTeX-Seitenzähler wird dabei angehalten.
		\item Inhalte können nach und nach Erscheinen oder nur zu bestimmten Zeiten sichtbar sein.
	\end{itemize}
	
	\begin{columns}
		\column{0.5\textwidth}{
			\begin{block}{Quelltext}
				\begin{lstlisting}
Erster Teil
\pause \\
Zweiter Teil
				\end{lstlisting}
			\end{block}}
		\column{0.5\textwidth}{\begin{block}{Vorschau}
				Erster Teil
				\pause \\
				Zweiter Teil
			\end{block}}
	\end{columns}
\end{frame}

\begin{frame}[fragile]
	\frametitle{\LaTeX\ Beamer -- Overlays -- Beispiele Quellcode}
	\begin{block}{Beispiele für Overlay Steuerung}
		\begin{lstlisting}
\begin{itemize}
  \visible<1>{\item Dieser Text erscheint nur auf Overlay 1.}
  {\color<1-3>{red}{\item Dieser Text ist auf Overlays 1 bis 3 rot.}}
  {\color<2->{blue}{\item Dieser Text ist ab Overlay 2 blau.}}
  \only<-3>{\item Dieser Text erscheint nur bis Overlay 3.}
  \textbf<1,3,5>{\item Dieser Text erscheint auf Overlays 1, 3 und 5 im Fettdruck.}
  \alt<2>{\item Dieser Text erscheint nur auf Overlay 2.}{\item Sonst erscheint dieser Text.}
\end{itemize}
		\end{lstlisting}
	\end{block}
\end{frame}

\begin{frame}[fragile]
	\frametitle{\LaTeX\ Beamer -- Overlays -- Beispiele Quellcode}
	\begin{flushright}
		Overlay \only<1>{1}\only<2>{2}\only<3>{3}\only<4>{4}\only<5>{5} / 5
	\end{flushright}
	\begin{itemize}
		\visible<1>{\item Dieser Text erscheint nur auf Overlay 1.}
		{\color<1-3>{red}{\item Dieser Text ist auf Overlays 1 bis 3 rot.}}
		{\color<2->{blue}{\item Dieser Text ist ab Overlay 2 blau.}}
		\only<-3>{\item Dieser Text erscheint nur bis Overlay 3.}
		\textbf<1,3,5>{\item Dieser Text erscheint auf Overlays 1, 3 und 5 im Fettdruck.}
		\alt<2>{\item Dieser Text erscheint nur auf Overlay 2.}{\item Sonst erscheint dieser Text.}
	\end{itemize}
\end{frame}
\begin{frame}
	\begin{block}{\huge Aufgabe 10 (Präsentationen)}
		Erstellen Sie eine einfache Präsentation mit mehreren Folien. Benutzen Sie dazu das \LaTeX{} \keyword{beamer}-Paket. Folgende Merkmale sollen in der Präsentation enthalten sein: Titelseite, Aufzählungen, Blöcke, zwei Spalten auf einer Folie und Overlays.
	\end{block}
\end{frame}


%%%%%%%%%%%%%%%%%%%%%%%%%%%%%%%%%%%%%%%%%%%%%%%%%%%%%%%%%%%%%%%%%%%%%%%%%%%%%
% S C H Ü L E R - V E R S I O N
%%%%%%%%%%%%%%%%%%%%%%%%%%%%%%%%%%%%%%%%%%%%%%%%%%%%%%%%%%%%%%%%%%%%%%%%%%%%%
%
\begin{frame}
	\frametitle{Über \LaTeX}

	\begin{itemize}
		\itemra ein Textsatzsystem
        \itemra \TeX{} ist eine Programmiersprache von Donald E. Knuth
        \itemra \LaTeX{} ist eine Sammlung von Makros für \TeX{} von Leslie Lamport
		\itemra Dokumente werden aus Quelltext erzeugt
		\itemra Quelltext ist ``logisches Markup'' (wie HTML)
		\itemra kein WYSIWYG (\textbf What\textbf You\textbf See\textbf Is\textbf What\textbf You\textbf Get)
		\itemra standardisierte Dokumente
		\itemra einfacher Formelsatz
		\itemra kostenlos und für alle gängigen Betriebssysteme erhältlich
		\itemra Standard für wissenschaftliche Veröffentlichungen
	\end{itemize}
\end{frame}

\begin{frame}
	\frametitle{Woher bekommt man \LaTeX?}
	
	\begin{itemize}
		\item offizielle Homepage: \url{http://www.latex-project.org}
		\item verschiedene Distributionen verfügbar:
		\begin{center}
			\begin{tabular}{rl}
				{\TUgreen TeX Live} & \url{http://tug.org/texlive/} \\
					& Installer oder direkt von USB-Stick startbar\\
				{\TUgreen Mik TeX} & \url{http://www.miktex.org} \\
					& für Windows-Systeme; Installer oder portable Version\\
				{\TUgreen Mac TeX} & \url{http://tug.org/mactex/} \\
					& TeX Live für Apple Mac-Systeme
			\end{tabular}
		\end{center}
		\item detailierte Beschreibung des Installationsvorgangs für \TeX Live:
		\begin{center}
			\url{http://www.dante.de/tex.html}
		\end{center}
	\end{itemize}
\end{frame}

\begin{frame}
	\frametitle{Grundaufbau einer \LaTeX-Datei}
	
	\begin{itemize}
		\item Präambel
			\begin{itemize}
				\item[$\to$] Stilvorgaben (Schriftart, Schriftgröße, Seitenränder, Kopf-/Fußzeile, usw.)
				\item[$\to$] Pakete mit zusätzlichen Funktionen laden (z. B. Farben, Verlinkung, usw.)\\~
			\end{itemize}
		\item Dokument
			\begin{itemize}
				\item[$\to$] Inhaltsverzeichnis
				\item[$\to$] Kapitel und Überschriften
				\item[$\to$] Text und Bilder
				\item[$\Ra$] der gesamte Inhalt
			\end{itemize}
	\end{itemize}
\end{frame}

\begin{frame}[fragile]
	\frametitle{\LaTeX: Ablauf der Dokumenterstellung}

\begin{center}
	\begin{tikzpicture}%[nodes=draw]
	\node[outer sep=5pt, anchor=base]            (a) {~};
	\node[outer sep=5pt, anchor=base] at (5,0) (b) {~};
	\node[outer sep=5pt, anchor=base] at (7,0) (c) {~};
	\node[outer sep=5pt, anchor=base] at (12,0) (d) {~};
	\node[outer sep=5pt, anchor=base] at (14,0) (e) {~};
	\node[outer sep=5pt, anchor=base] at (19,0) (f) {~};
	\path let \p0=($(a.west)-(b.east)$) in
	  (a) --
	  node [
	    pos=.5,
	    auto=false,
	    shape=single arrow,
	    anchor=tail,
	    draw=none,
	%    line join=round,
	    xshift=-veclen(\p0)/2,
	    minimum height=veclen(\p0)-\pgfkeysvalueof{/pgf/outer xsep},
	    fill=TUgreen,
	    decorate, decoration={
	      name=random steps,
	      segment length=+.5pt,
	      amplitude=.5pt}
	  ] {Editor}
	  (b);
\node[align=left] at (2,-2) {zur Eingabe des\\ sog. \textbf{Quellcode}\\(bei uns: Kile)};	  
	  
	\path let \p0=($(c.west)-(d.east)$) in
	  (c) --
	  node [
	    pos=.5,
	    auto=false,
	    shape=single arrow,
	    anchor=tail,
	    draw=none,
	%    line join=round,
	    xshift=-veclen(\p0)/2,
	    minimum height=veclen(\p0)-\pgfkeysvalueof{/pgf/outer xsep},
	    fill=TUgreen,
	    decorate, decoration={
	      name=random steps,
	      segment length=+.5pt,
	      amplitude=.5pt}
	  ] {\LaTeX-Compiler}
	  (d);
\node[align=left] at (9.3,-2) {übersetzt den Quellcode\\ in ein Dokumentformat\\ (z. B. PDF)};	 

	\path let \p0=($(e.west)-(f.east)$) in
	  (e) --
	  node [
	    pos=.5,
	    auto=false,
	    shape=single arrow,
	    anchor=tail,
	    draw=none,
	%    line join=round,
	    xshift=-veclen(\p0)/2,
	    minimum height=veclen(\p0)-\pgfkeysvalueof{/pgf/outer xsep},
	    fill=TUgreen,
	    decorate, decoration={
	      name=random steps,
	      segment length=+.5pt,
	      amplitude=.5pt}
	  ] {Betrachter}
	  (f);
\node[align=left] at (16.6,-2) {zeigt das Dokument an};
	\end{tikzpicture}
\end{center}

\end{frame}



\begin{frame}
	\frametitle{Kile - \LaTeX Entwicklungsumgebung}
	
	\begin{itemize}
			\item erleichtert das Bearbeiten von mehreren Dokumenten
			\item Autovervollständigung
			\item Hilfe zum Kompilieren der Dokumente
			\item Syntaxhervorhebung
			\item Rechtschreibprüfung
			\item Code-Faltung
	\end{itemize}
\end{frame}

\begin{frame}[plain]
	\includegraphics[width=\textwidth]{screenshots/kile-start.png}
\end{frame}

\begin{frame}[plain,noframenumbering]
	\includegraphics[width=\textwidth]{screenshots/kile-new.png}
\end{frame}

\begin{frame}[plain,noframenumbering]
	\includegraphics[width=\textwidth]{screenshots/kile-new-doc.png}
\end{frame}

\begin{frame}[plain,noframenumbering]
	\includegraphics[width=\textwidth]{screenshots/kile-new-save-as.png}
\end{frame}

\begin{frame}[plain,noframenumbering]
	\includegraphics[width=\textwidth]{screenshots/kile-new-save-as-create-folder.png}
\end{frame}

\begin{frame}[plain,noframenumbering]
	\includegraphics[width=\textwidth]{screenshots/kile-new-save-as-2.png}
\end{frame}

\begin{frame}[plain,noframenumbering]
	\includegraphics[width=\textwidth]{screenshots/kile-quickbuild.png}
\end{frame}

\begin{frame}[plain,noframenumbering]
	\includegraphics[width=\textwidth]{screenshots/kile-quickbuild-2.png}
\end{frame}

\begin{frame}[fragile]
	\frametitle{Hello World!}
	
	\latexBeispielDatei{Ein erstes \LaTeX-Dokument}{examples/hello_world/hello_world}
\end{frame}

\begin{frame}[fragile]
	\frametitle{Generierung einer PDF-Datei}
    Um eine PDF-Datei aus einem \LaTeX{} Dokument \dateiname{datei.tex} zu erzeugen, muss der \LaTeX-Compiler aufgerufen werden:\\
	\kommandozeile{pdflatex datei.tex} \\[1cm]
Alternativ kann man wie folgt vorgehen:
	\begin{enumerate}
		\item \kommandozeile{latex datei.tex}\\
		Dieser Befehl ruft den \LaTeX-Compiler auf und erzeugt die DVI-Datei \dateiname{datei.dvi}.
		\item \kommandozeile{dvips datei.dvi}\\
		Konvertiert die \dateiname{datei.dvi} in eine Post-Script Datei \dateiname{datei.ps}.
		\item \kommandozeile{ps2pdf} oder direkt \kommandozeile{dvipdf}\\
		Erzeugt eine PDF-Datei \dateiname{datei.pdf}.
	\end{enumerate}
\end{frame}

\begin{frame}
	\frametitle{Eingabe eines einfachen Textes}
	
	\latexBeispielDateiKlein{Quelltext}{examples/Eingabe_Quelltext/Eingabe_Quelltext}
\end{frame}

\begin{frame}
	\frametitle{Befehle und Umgebungen}
	\begin{itemize}
        \item \emphword{Befehle} haben in \LaTeX{} die Form \befehl{befehl}
		\item Beispiel: \befehl{textit\{kursiv\}} schaltet auf {\it Kursivschrift} um
		\item Manche Befehle benötigen zusätzliche Parameter, z. B.\\
		\befehl{textbf\{Text, der fett gedruckt sein soll\}}\\
		oder optionale Parameter, z. B.\\
		\befehl{documentclass[a4paper]\{article\}}\\[1cm]
		\item \emphword{Umgebungen} beeinflussen den gesamten enthaltenen Text:\\
		\befehl{begin\{center\}} \\
		~~~~zentrierter Text\\
		\befehl{end\{center\}}\\
		Auch hier ist die Angabe von Parametern möglich.
	\end{itemize}
\end{frame}

%\begin{frame}
	\frametitle{Spezielle Zeichen}
	\begin{itemize}
		\item bei Verwendung des ASCII-Zeichensatzes: Buchstaben ohne Umlaute, Zahlen und einige Sonderzeichen
		\item manche Zeichen sind \LaTeX-Steuerzeichen und daher reserviert (\$, \_, \{, \},\textbackslash )
        \item solche Sonderzeichen können durch \textbackslash{} maskiert werden: \\[0.5cm]
		\begin{center}
			\begin{tabular}{|cc|cc|} \hline
				\$ & \textbackslash\$ & \% & \textbackslash\% \\
				\{ & \textbackslash\{ & \} & \textbackslash\} \\
				\# & \textbackslash\# & \_ & \textbackslash\_ \\
				\& & \textbackslash\& & & \\ \hline
			\end{tabular}
		\end{center}
	\end{itemize}
\end{frame}

\begin{frame}[fragile]
	\frametitle{Deutsche Texte - Sprachpakete}
    Ohne weitere Angaben nimmt \LaTeX{} an, dass der eingegebene Text in englischer Sprache ist. Daher muss ggf. ein zusätzliches Sprachpaket eingebunden werden:
	\begin{center}
		\begin{block}{Beispiel-Header: deutsches Sprachpaket}
			\begin{lstlisting}
\documentclass[a4paper]{article}      % DIN-A4 Papierformat
\usepackage[ngerman]{babel}           % deutsche Benennung
\usepackage[utf8]{inputenc}
\begin{document}
  ...
\end{document}
			\end{lstlisting}
		\end{block}
	\end{center} \vspace{-1cm}
	\begin{itemize}
		\item \keyword{babel} sorgt für Unterstützung anderer Sprachen (Formate, Umlaute, Benennungen, Silbentrennung)
		\item \keyword{inputenc} unterstützt die direkte Eingabe von Zeichen über die Tastatur
	\end{itemize}
\end{frame}


\begin{frame}[fragile]
	\frametitle{Deutsche Texte -- Umlaute und Anführungszeichen}
	
	Eingabe deutscher Texte:
	\begin{itemize}
		\item Umlaute: \befehl{"a}, \befehl{"o}, \befehl{"u}, \befehl{ss} für ä, ö, ü, ß
		\item Anführungszeichen: \befehl{glq}, \befehl{grq} bzw. \befehl{glqq}, \befehl{grqq} für \glq einfache\grq~ bzw. \glqq doppelte\grqq~ Anführungszeichen
	\end{itemize}
	\vfill
	
	\latexBeispielDirekt{Beispiel: deutsche Umlaute}{examples/Deutsche_Umlaute/Deutsche_Umlaute}
	\vfill
\end{frame}

\begin{frame}[fragile]
	\frametitle{Deutsche Texte - Silbentrennung}
	\begin{itemize}
		\item erfolgt automatisch
		\item mögliche Trennstellen können  durch \befehl{-} auch angegeben werden, z. B.\\
		\lstinline$Donau\-dampf\-schiff\-fahrts\-gesell\-schaft$\\
		oder für das gesamte Dokument in der Präambel:
		\lstinline$\hypenation{Donau\-dampf\-schiff\-fahrts\-gesell\-schaft}$
	\end{itemize}

\end{frame}

%\begin{frame}[fragile]
	\frametitle{Textformatierung -- Schriftstil}
	
	\begin{center}
		\begin{tabular}{l|ll|l}
			Familie & \multicolumn{2}{c|}{Befehle} & Beispiel \\ \hline
			normal (mit Serifen) & \befehl{rmfamily} & \befehl{textr} & \textrm{normal} \\
			serifenfrei & \befehl{sffamily} & \befehl{textsf} & \textsf{serifenfrei} \\
			Schreibmaschine & \befehl{ttfamily} & \befehl{texttt} & \texttt{Schreibmaschine} \\
		\multicolumn{4}{c}{~} \\
			Varianten & \multicolumn{2}{c|}{Befehle} & Beispiel \\ \hline
			aufrecht & \befehl{upshape} & \befehl{textup} & \textup{aufrecht} \\
			italic & \befehl{itshape} & \befehl{textit} & \textit{italic}\\
			Kapitälchen & \befehl{scshape} & \befehl{textsc} & \textsc{Kapitälchen}\\
			fett & \befehl{bfseries} & \befehl{textbf} & \textbf{fett}\\
			unterstrichen & ~ & \befehl{underline} & \underline{unterstrichen}
		\end{tabular}
	\end{center}
\end{frame}

\begin{frame}[fragile]
	\frametitle{Textformatierung -- Schriftgröße}
	\begin{center}
		\begin{tabular}{ll}
		\befehl{tiny} & \tiny{winzig} \\
		\befehl{small} & \small{klein} \\
		\befehl{footnotesize} & {\fontsize{14}{14}\selectfont{}Fußnotengröße}\\
		\befehl{normalsize} & normale Größe\\
		\befehl{large} & {\fontsize{25}{25}\selectfont{}groß} \\
		\befehl{Large} & {\fontsize{30}{30}\selectfont{}größer} \\
		\befehl{huge} & {\fontsize{40}{40}\selectfont{}riesig} \\
		\befehl{Huge} & {\fontsize{50}{50}\selectfont{}Riesig}
		\end{tabular}
	\end{center}
	\vfill
	\begin{itemize}
		\item für einige Wörter: \lstinline${\huge riesig}$
		\item für ganze Absätze: \lstinline$\begin{tiny} ... \end{tiny}$
		\item alternativ: punktgenau durch \lstinline${\fontsize{40}{48}\selectfont{}Test}$\\
		das erste Argument gibt die Schriftgröße an, das zweite den Grundlinienabstand
	\end{itemize}
\end{frame}



%\begin{frame}[fragile]
	\frametitle{Aufzählungen}
	\vspace{-0.5cm}
	drei Grundarten von Aufzählungen: \\
	\begin{tabular}{rl}
		\emphkeyword{itemize} & einfache Aufzählung\\
		\emphkeyword{enumerate} & nummerierte Aufzählung\\
		\emphkeyword{description} & Beschreibung
	\end{tabular}\par \vfill
	
	\begin{columns}[T]
		\begin{column}{0.31\textwidth}
			\begin{block}{\tt\bfseries itemize}
				\begin{itemize}
					\item A \item B \item C
				\end{itemize} 
				\begin{lstlisting}
\begin{itemize}
  \item A
  \item B
  \item C
\end{itemize}
				\end{lstlisting}
			\end{block}
		\end{column}
		\begin{column}{0.31\textwidth}
			\begin{block}{\tt\bfseries enumerate}
				\begin{enumerate}
					\item A \item B \item C
				\end{enumerate} 
				\begin{lstlisting}
\begin{enumerate}
  \item A
  \item B
  \item C
\end{enumerate}
				\end{lstlisting}
			\end{block}
		\end{column}
		\begin{column}{0.31\textwidth}
			\begin{block}{\tt\bfseries description}
				\begin{description}
					\item[A] \ldots \item[B] \ldots \item[C] \ldots
				\end{description} 
				\begin{lstlisting}
\begin{description}
  \item[A] \ldots
  \item[B] \ldots
  \item[C] \ldots
\end{description}
				\end{lstlisting}
			\end{block}
		\end{column}
	\end{columns}
\end{frame}

\begin{frame}[fragile]
	\frametitle{Verschachtelte Aufzählungen}
	\vspace{-0.9cm}
	\latexBeispielDirekt{Beispiel: verschachtelte Aufzählung}{examples/Verschachtelte_Aufzaehlung/Verschachtelte_Aufzaehlung}
\end{frame}
%
\begin{frame}[fragile]
	\frametitle{Mathematik -- Formeln im Fließtext}
	\begin{itemize}
        \item Formeln müssen in \LaTeX{} markiert werden, damit sie korrekt interpretiert werden
		\item im Mathematik-Modus werden Leerzeichen ignoriert und Buchstabenketten als einzelne Zeichen betrachtet
	\end{itemize}
	\vfill
	\latexBeispielDirekt{Beispiel: Formeln im Fließtext}{examples/Formeln_im_Fliesstext/Formeln_im_Fliesstext}
\end{frame}

\begin{frame}[fragile]
	\frametitle{Mathematik -- Formeln in eigenem Absatz}
	\begin{itemize}
		\item soll eine Formel abgesetzt dargestellt werden, so benutzt man die \keyword{displaymath}-Umgebung oder die Kurzschreibweise \lstinline$\[...\]$
	\end{itemize}
	\vfill
	\latexBeispielDirekt{Beispiel: Formeln in eigenem Absatz}{examples/Formeln_eigener_Absatz/Formeln_eigener_Absatz}
\end{frame}

\begin{frame}[fragile]
	\frametitle{Mathematik -- nummerierte Gleichungen}
	\begin{itemize}
		\item Gleichungen können mit der \keyword{equation}-Umgebung automatisch fortlaufend nummeriert werden
		\item mittels der \keyword{align}-Umgebung können mehrzeilige Formeln nummeriert und ausgerichtet werden
		\item \befehl{nonumber} unterdrückt in der \keyword{align}-Umgebung die Nummerierung einer Zeile
	\end{itemize}
	\vfill
	\latexBeispielDirektKlein{Beispiel: mehrzeilige Gleichungen}{examples/Formeln_mehrzeilig/Formeln_mehrzeilig}
\end{frame}




\begin{frame}
	\frametitle{Literaturangaben}
	\begin{itemize}
		\item Thomas Rohkämper (TU Dortmund): \textit{Einführung in \LaTeX}, Sommersemester 2014
		\begin{itemize}
			\item Deutschsprachige Anwendervereinigung TeX e. V.\\
			\href{http://www.dante.de}{http://www.dante.de}
			\item Helmut Kopka. \textit{LATEX, Bd. 1: Einführung}. Pearson Studium, 2003.
			\item Joachim Schlosser. \textit{Wissenschaftliche Arbeiten schreiben mit \LaTeX: Leitfaden für Einsteiger}. mitp, 2009.
            \item Petra Schlager und Manfred Thibud. \textit{Wissenschaftlich mit \LaTeX{} arbeiten}. Pearson Studium, 2005.
			\item Wikibooks: LaTeX\\
			\href{http://en.wikibooks.org/wiki/LaTeX}{http://en.wikibooks.org/wiki/LaTeX}
		\end{itemize}
    \item Thomas Rohkämper (TU Dortmund): \textit{Einführung in \LaTeX{} -- Präsentationen mit \textit{beamer}}, Sommersemester 2014
	\end{itemize}
\end{frame}

\end{document}
