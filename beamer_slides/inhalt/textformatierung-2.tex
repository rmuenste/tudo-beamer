\begin{frame}[fragile]
	\frametitle{Textformatierung -- minipages}
	\begin{itemize}
		\item die \keyword{minipage}-Umgebung wird genutzt, wenn in Boxen komplexere Inhalte  dargestellt werden sollen
		\item es muss die Breite angegeben werden
	\end{itemize}
	\latexBeispielDirekt{Beispiel: minipages}{examples/minipages/minipages.tex}
\end{frame}

\begin{frame}[fragile]
	\frametitle{Textformatierung -- Absatzausrichtung} \vspace{-0.7cm}
	\latexBeispielDirekt{Beispiel: Absatzausrichtung}{examples/absatzausrichtung/absatzausrichtung.tex}
	\vfill
\begin{itemize}
	\item Standard: Blocksatz
	\item alternativ kann man auch die Befehle \befehl{raggedleft}, \befehl{raggedright} und \befehl{centering} verwenden (statt den obigen Umgebungen)
\end{itemize}
\end{frame}

\begin{frame}[fragile]
	\frametitle{Textformatierung - Zitate und Gedichte}
	\latexBeispielDirektKlein{Beispiel: Zitate und Gedichte}{examples/Zitate_und_Gedichte/Zitate_und_Gedichte.tex}
\end{frame}

\begin{frame}[fragile]
	\frametitle{Text ohne Formatierung}
	\begin{itemize}
		\item um Programmcode oder Beispiele ohne Formatierung auszugeben, gibt es die \keyword{verbatim}-Umgebung
		\item innerhalb eines Textes kann man den Befehl \befehl{verb} nutzen; das erste Zeichen nach dem Befehl wird dann als Endmarker interpretiert
	\end{itemize}\vfill
	\latexBeispielDirekt{Beispiel: \keyword{verbatim}-Umgebung}{examples/verbatim/verbatim.tex}
\end{frame}

\begin{frame}[fragile]
	\frametitle{Randnotizen und Fußnoten}
	\latexBeispielDatei{Beispiel: Randnotiz und Fußnote}{examples/Randnotiz_und_Fussnote/Randnotiz_und_Fussnote}
\end{frame}