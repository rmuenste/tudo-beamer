\begin{frame}[fragile]
  \frametitle{Inhaltsverzeichnis ausgeben}
  \begin{itemize}
    \item um Standardverzeichnisse auszugeben, muss man einfach die entsprechenden Schlüsselwörter angeben:
      \begin{itemize}
        \item \befehl{tableofcontents}
        \item \befehl{listoffigures}
        \item \befehl{listoftables}
      \end{itemize}
    \item die maximale Tiefe des Inhaltsverzeichnisses kann mit folgendem Befehl geändert werden:
      \begin{itemize}
        \item \befehl{setcounter\{tocdepth\}\{1\}}
      \end{itemize}
    \item manuelle Einträge können so eingefügt werden:
      \begin{itemize}
        \item \befehl{addcontentsline\{toc\}\{subsection\}\{Titel\}}
      \end{itemize}
    \item mögliche Werte für das Zielverzeichnis:
      \begin{center}
        \begin{tabular}{lll}
          \emphkeyword{toc} & \textit{table of contents} &\keyword{chapter}, \keyword{section}, \keyword{subsection} \\ && \keyword{subsubsection}, \keyword{paragraph} \\
          \emphkeyword{lof} & \textit{list of figures} & \keyword{figure} \\
          \emphkeyword{lot} & \textit{list of tables} & \keyword{table}
        \end{tabular}
      \end{center}
  \end{itemize}
\end{frame}

