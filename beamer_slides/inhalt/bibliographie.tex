\begin{frame}[fragile]
	\frametitle{Bibliographie}
	\begin{itemize}
        \item \LaTeX{} stellt mit BibTeX ein sehr mächtiges System zur Verwaltung von Literaturverweisen bereit.
		\item Um BibTeX nutzen zu können, muss zunächst eine Datenbank angelegt werden.
		\item Die Datenbank ist eine einfache Textdatei mit Einträgen für die verschiedenen zitierten Quellen.
		\item allgemeiner Aufbau eines Eintrags:
		\begin{itemize}
			\item \keyword{@literaturtyp\{kennung, name1=''Wert1'', name2=''Wert2'', ...\}}
		\end{itemize}
		\begin{center}
			\begin{tabular}{rl}
				\emphkeyword{literaturtyp} & spezifiziert die Art der Quelle\\
				& unterschiedliche Eintragstypen erfordern unterschiedliche\\
				& Angaben zur Quelle \\[0.2cm]
				\emphkeyword{kennung} & damit kann auf den Eintrag referenziert werden\\[0.2cm]
				\emphkeyword{name=''wert''} & Zuweisung von Werten an die verschiedenen Felder
			\end{tabular}
		\end{center}
	\end{itemize}
\end{frame}

\begin{frame}[fragile]
	\frametitle{Bibliographie -- Unterstützte Literatur-Typen}
	\begin{center}
		\begin{tabular}{rl}
			\emphkeyword{article} & Veröffentlichung in einer Zeitschrift\\
			\emphkeyword{book} & Buch\\
			\emphkeyword{booklet} & Buch ohne Verleger\\
			\emphkeyword{inbook} & Teil eines Buches\\
			\emphkeyword{incollection} & Teil einer Buchreihe\\
			\emphkeyword{inproceedings} & Teil einer Veröffentlichung zu einer Konferenz\\
			\emphkeyword{manual} & Technische Dokumentation \\
			\emphkeyword{masterthesis} & Masterarbeit\\
			\emphkeyword{phdthesis} & Doktorarbeit\\
			\emphkeyword{proceedings} & Veröffentlichung zu einer Konferenz\\
			\emphkeyword{techreport} & Technischer Bericht\\
			\emphkeyword{unpublished} & unveröffentlicht\\
			\emphkeyword{misc} & falls alles andere nicht passt
		\end{tabular}
	\end{center}
\end{frame}	

\begin{frame}[fragile]
	\frametitle{Bibliographie -- Unterstützte Felder}
	\vspace{-0.7cm}
	\begin{center}
		\begin{tabular}{rl}
			\emphkeyword{address} & die Adresse des Verlags oder einer anderen Institution\\
			\emphkeyword{annotate} & Anmerkungen \\
			\emphkeyword{author} & Namen der Autoren (in BibTeX Format)\\
			& \tabitem mehrere Namen werden durch \keyword{AND} getrennt \\
			& \tabitem zwei Möglichkeiten Namen zu schreiben:\\
			& ~~~ \keyword{Donald E. Knuth} \textit{oder} \keyword{Knuth, Donald E.}\\
			\emphkeyword{booktitle} & Titel des Buchs \\
			\emphkeyword{chapter} & Kapitel- oder Abschnitt-Nummer\\
			\emphkeyword{crossref} & Datenbank-Schlüssel\\
			\emphkeyword{edition} & Auflage (z. B. eines Buchs)\\
			\emphkeyword{editor} & Namen der Editoren (analog dem \keyword{author}-Feld)\\
			\emphkeyword{howpublished} & Veröffentlichungsart \\
			\emphkeyword{institution} & fördernde Institution eines technischen Reports\\	
			\emphkeyword{journal} & Zeitschriftenname (häufig abgekürzt)
		\end{tabular}
	\end{center}
\end{frame}

\begin{frame}[fragile]
	\frametitle{Bibliographie -- Unterstützte Felder}
	\vspace{-0.7cm}
	\begin{center}
		\begin{tabular}{rl}
            \emphkeyword{key} & Feld zur Sortierung und Erstellung von Labels\\
			\emphkeyword{month} & Monat der Veröffentlichung/Erscheinung\\
			\emphkeyword{note} & zusätzliche Information\\
			\emphkeyword{number} & Nummer einer Zeitschrift, eines Reports oder eines Bandes\\
			\emphkeyword{organization} & fördernde Organisation\\
			\emphkeyword{pages} & Seitenzahlen oder Seitenzahlbereich (z. B. \keyword{7-33})\\
			\emphkeyword{publisher} & Name des Verlags\\
			\emphkeyword{school} & Name der ``Schule'', an der eine Abschlussarbeit geschrieben wurde\\
			\emphkeyword{series} & Name einer Reihe\\
			\emphkeyword{title} & Titel des Werkes\\
			\emphkeyword{type} & Typ eines technischen Reports\\
			\emphkeyword{volume} & Band einer Zeitschrift oder eines Buches\\
			\emphkeyword{year} & Jahr der Veröffentlichung/Erscheinung
		\end{tabular}
	\end{center}
\end{frame}

\begin{frame}[fragile]
	\frametitle{Bibliographie einbinden}
	\begin{itemize}
		\item verschiedene Felder sind bei verschiedenen Literaturtypen vorgeschrieben
		\item es gibt noch weitere Felder wie \keyword{ISBN}, \keyword{doi}, \keyword{abstract}, ...
		\item Sonderzeichen und Umlaute müssen speziell maskiert werden\\[0.5cm]
		\item BibTeX muss eigens aufgerufen werden:\\
		\kommandozeile{latex beispiel}\\
		\kommandozeile{bibtex beispiel}\\
		\kommandozeile{latex beispiel}\\
		\kommandozeile{latex beispiel}
		\item BibTeX prüft die Syntax der Einträge und erzeugt eine Datei, die in das \LaTeX-Dokument eingebunden werden kann
        \item nach Aufruf von BibTeX muss \LaTeX{} die Datei zwei mal kompilieren, damit alle Referenzen korrekt gesetzt werden
	\end{itemize}
\end{frame}

\begin{frame}[fragile]
	\frametitle{Bibliographie einbinden}
	\begin{itemize}
		\item um die Bibliographie anzuzeigen, verwendet man den folgenden Befehl:\\
		\befehl{bibliography\{literatur\}}
		\item \keyword{literatur} bezeichnet dabei die Datei mit den entsprechenden Literaturangaben
		\item aufgelistet werden alle Quellen, die im Dokument zitiert wurden (mittels dem Befehl \befehl{cite\{kennung\}})
		\item durch \befehl{nocite\{kennung\}} wird auch der \keyword{kennung} entsprechende Eintrag aufgeführt, auch wenn er nicht zitiert wurde
		\item die Formatierung hängt von \befehl{bibliographystyle} ab
		\item es gibt verschiedene Pakete zur Gestaltung der Literaturverweise, z. B. \keyword{natbib}
	\end{itemize}
\end{frame}

\begin{frame}[fragile]
	\frametitle{Bibliographie mit \emphkeyword{natbib}}
	\vspace{-1cm}
	\begin{itemize}
		\item \keyword{natbib} wird durch \befehl{usepackage\{natbib\}} (in der Präambel) eingebunden
		\item als Format wird dann \keyword{plainnat} gewählt:
		\befehl{bibliographystyle\{plainnat\}}
		\item \keyword{natbib} unterstützt mehrere zusätzliche Literaturangaben wie
		\begin{center}
			\begin{tabular}{rl}
				\emphkeyword{ISBN} & ISBN-Nummer eines Buches\\
				\emphkeyword{ISSN} & ISSN-Nummer einer Zeitschrift\\
				\emphkeyword{URL} & Internet-Adresse für Online-Dokumente\\
				\emphkeyword{DOI} & Digital Object Identifier
			\end{tabular}
		\end{center}
	\end{itemize}
	
	\begin{columns}[c]
		\begin{column}{0.54\textwidth}
			\begin{block}{Zitieren mit \keyword{natbib}}
			\lstinputlisting{examples/NatBib/NatBib}
			\end{block}
		\end{column}
		\begin{column}{0.44\textwidth}
			\shadowbox{\begin{minipage}{0.99\textwidth} Knuth et al. (1993)\\(Knuth et al., 1993)\\(siehe Knuth et al., 1993, S. 13)\\Knuth et al.\\1993 \end{minipage}}
		\end{column}
	\end{columns}
\end{frame}
