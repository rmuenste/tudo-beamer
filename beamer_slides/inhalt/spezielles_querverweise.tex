\begin{frame}[fragile]
	\frametitle{Querverweise}
	\begin{itemize}
		\item Befehle zur Benutzung von Querverweisen:
		\begin{center}
			\begin{tabular}{ll}
				\befehl{label\{marker\}} & Verbindet die momentane Textstelle mit \keyword{marker}. So kann\\
				& von einer anderen Textstelle aus auf diese Stelle verwiesen\\
				& werden.\\[0.5cm]
				\befehl{ref\{marker\}} & Erzeugt einen Querverweis auf eine Stelle, die zuvor mittels\\
				& \keyword{label} gekennzeichnet wurde. Der Querverweis gibt die\\
				& Gliederungsnummer der betreffenden Textstelle an.\\[0.5cm]
				\befehl{pageref\{marker\}} & Wie \befehl{ref}, gibt jedoch die Seitennummer der Textstelle\\
				& zurück.
			\end{tabular}
		\end{center}
	\end{itemize}
\end{frame}

\begin{frame}[fragile]
	\frametitle{Querverweise}
	\begin{itemize}
		\item Durch ein Präfix (optional) kann angegeben werden, was referenziert wird, z. B.:\\
		\befehl{label\{sec:Querverweise\}}
		\item mögliche Präfixe sind:
		\begin{center}
			\begin{tabular}{rl}
				\emphkeyword{chap} & chapter\\
				\emphkeyword{sec} &  section\\
				\emphkeyword{fig} & figure\\
				\emphkeyword{tab} & table\\
				\emphkeyword{eq} & equation\\
				\emphkeyword{lst} & listing
			\end{tabular}
		\end{center}
		\item in \keyword{figure}- und \keyword{table}-Umgebung muss das Label innerhalb des \befehl{caption}-Befehls gesetzt werden
	\end{itemize}
\end{frame}

\begin{frame}[fragile]
	\frametitle{Querverweise -- Beispiel}
	
	\latexBeispielDirekt{Querverweise -- Beispiel}{examples/Querverweise/Querverweise.tex}
\end{frame}
