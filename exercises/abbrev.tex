% ------------------------------------------------------------------------------------
% D I P L O M A R B E I T     FRIEDRICH-ALEXANDER-UNIVERSITAET ERLANGEN-NUERNBERG
% ------------------------------------------------------------------------------------
%     erstellt von Christopher Basting, 23. August 2011
% ------------------------------------------------------------------------------------
% abbrev.tex
% ------------------------------------------------------------------------------------

\renewcommand{\epsilon}{\varepsilon}

\renewcommand{\d}{{\rm d}} % d fuer dx bei Ableitungen / Integralen
\newcommand{\gdw}{\Leftrightarrow} % <=>
\newcommand{\x}{\cdot}
\newcommand{\ma}{\left( \begin{array}{*{19}{c}}} % Matrix oeffnen
\newcommand{\me}{\end{array}\right)} % Matrix schliessen
\newcommand{\mA}{\left[ \begin{array}{*{19}{c}}} % Matrix oeffnen
\newcommand{\mE}{\end{array}\right]} % Matrix schliessen
\newcommand{\C}{\mathds{C}}
\newcommand{\E}{\mathds{1}}
\newcommand{\F}{\mathds{F}}
\newcommand{\K}{\mathds{K}}
\newcommand{\N}{\mathds{N}}
\newcommand{\Q}{\mathds{Q}}
\newcommand{\R}{\mathds{R}}
\newcommand{\Z}{\mathds{Z}}
\newcommand{\id}{\mathds{1}}
\renewcommand{\d}{\,{\rm d}}
\newcommand{\D}{{\rm D}}
\newcommand{\richtig}{\checkmark}
\newcommand{\falsch}{\lightning}
\newcommand{\ra}{\rightarrow}
\newcommand{\Ra}{\Rightarrow}
\newcommand{\la}{\leftarrow}
\newcommand{\La}{\Leftarrow}

\newcommand{\Limn}{\lim\limits_{n \rightarrow \infty}}
\newcommand{\Limh}{\lim\limits_{h \rightarrow 0}}
\newcommand{\Limoh}{\lim\limits_{h \searrow 0}}
\newcommand{\Limuh}{\lim\limits_{h \nearrow 0}}

\renewcommand{\Re}{{\rm Re}}
\renewcommand{\Im}{{\rm Im}}
\renewcommand{\i}{{\vec i}}

\newcommand{\abbildung}[3]{#1 \, : \, #2 \rightarrow #3}
\newcommand{\grad}{\nabla}

\renewcommand{\phi}{\varphi}
\newcommand{\cL}{\mathcal{L}}
\newcommand{\cO}{\mathcal{O}}

\newcommand{\eps}{\varepsilon}

\renewcommand{\vec}{\bold}
\newcommand{\dist}{{\rm dist}}

\newcommand{\mat}[1]{\textsf{\textbf{#1}}}

\usepackage{amscd,amssymb} 

\providecommand{\abs}[1]{\lvert#1\rvert}
\providecommand{\norm}[1]{\left\lVert#1\right\rVert}
\providecommand{\transpose}[1]{{#1}^\top}
\providecommand{\inversetranspose}[1]{{#1}^{-\top}}

\newcommand\Sum[2]{\sum\limits_{#1}^{#2}}
\newcommand{\sumin}{{\displaystyle\sum_{i=1}^n}}
\newcommand{\sumjn}{{\displaystyle\sum_{j=1}^n}}