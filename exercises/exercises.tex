\documentclass[fleqn,idxtotoc,bibtotoc,pointlessnumbers,ngerman]{scrartcl}

%%%%%%%%%%%%%%%%%%%%%%%%%%%%%%%%%%%%%%%%%%%%%%%%%%%%%%%%%%%%%%%%%%%%%%%%%%%%%%%%%%%%%%%%%%
\newcommand{\dozent}{Christopher Basting}
\newcommand{\veranstaltung}{Einf\"uhrung in \LaTeX}
\newcommand{\zeitraum}{M\"arz 2017}
%%%%%%%%%%%%%%%%%%%%%%%%%%%%%%%%%%%%%%%%%%%%%%%%%%%%%%%%%%%%%%%%%%%%%%%%%%%%%%%%%%%%%%%%%%

\usepackage[ngerman]{babel}

%\usepackage{gitinfo}
% default values
\providecommand{\dozent}{DOZENT}
\providecommand{\veranstaltung}{VERANSTALTUNG}
\providecommand{\zeitraum}{ZEITRAUM}

%%%%%%%%%%%%%%%%%%%%%%%%%%%%%%%%%%%%%%%%%%%%%%%%%%%%%%%%%%%%%%%%%%%%%%%%%%%%%%%%
\usepackage[utf8]{inputenc}
\usepackage[dvipsnames]{xcolor}
\newcommand{\changefont}[3]{\fontfamily{#1} \fontseries{#2} \fontshape{#3} \selectfont}

%%%%%%%%%%%%%%%%%%%%%%%%%%%%%%%%%%%%%%%%%%%%%%%%%%%%%%%%%%%%%%%%%%%%%%%%%%%%%%%%


\KOMAoptions{paper=a4}      % def. of pagesize
\KOMAoptions{twoside=false}
\KOMAoptions{fontsize=12pt}
\KOMAoptions{headinclude=true}
\KOMAoptions{footinclude=false}
\KOMAoptions{mpinclude=false}
\usepackage{geometry}
\geometry{inner=2cm, outer=2cm, top=2cm, bottom=2cm}
\geometry{includehead}
\geometry{headsep=2cm, footskip=0.0cm, headheight=1.9cm}
\usepackage{scrpage2}
\usepackage{booktabs}
\usepackage{dsfont}
\usepackage{enumitem}

%%%%%%%%%%%%%%%%%%%%%%%%%%%%%%%%%%%%%%%%%%%%%%%%%%%%%%%%%%%%%%%%%%%%%%%%%%%%%%%%

\raggedbottom % latex no longer streches vertical spaces s.th. text reaches the bottom
\usepackage{amsfonts,latexsym}
\usepackage{amsmath}
\usepackage{bm}
\usepackage{amscd,amssymb} % AMS-packages for theorem environments
\usepackage{fixmath} % upper case greek letters: normal style -> italic style
\usepackage{graphicx} % \includegraphics
\usepackage{eso-pic} % background images
\usepackage{pifont}

%%%%%%%%%%%%%%%%%%%%%%%%%%%%%%%%%%%%%%%%%%%%%%%%%%%%%%%%%%%%%%%%%%%%%%%%%%%%%%%%

\usepackage{hyperref}
\hypersetup{pdfview={Fit},pdfstartview={Fit}}
\hypersetup{breaklinks=true,colorlinks=true,pdfborder={0 0 0}}
\hypersetup{linkcolor=fau, anchorcolor=fau, citecolor=fau, filecolor=fau, urlcolor=fau}
\hypersetup{pdfauthor={\dozent},pdftitle={\veranstaltung}}
\hypersetup{pdftitle={\veranstaltung}}
%\hypersetup{pdftitle={\veranstaltung (Rev. \gitAbbrevHash)}}
%\hypersetup{pdfsubject={Revision \gitHash}}

%%%%%%%%%%%%%%%%%%%%%%%%%%%%%%%%%%%%%%%%%%%%%%%%%%%%%%%%%%%%%%%%%%%%%%%%%%%%%%%%

\definecolor{fau}{HTML}{84B819}

\newcounter{chapter}

\usepackage[amsmath,thmmarks,hyperref,thref]{ntheorem} % option hyperref erzeugt Fehler, 
\theoremstyle{break}
\theorembodyfont{\normalfont\slshape}
\theoremstyle{break}
\theorembodyfont{\normalfont\slshape}

%%%%%%%%%%%%%%%%%%%%%%%%%%%%%%%%%%%%%%%%%%%%%%%%%%%%%%%%%%%%%%%%%%%%%%%%%%%%%%%%

\setlength{\theorempreskipamount}{1.5em}
\theoremseparator{}
\theoremstyle{break}
\theoremsymbol{}
\theoremheaderfont{\normalfont\sffamily\bfseries\color{fau}}
\theorembodyfont{\sf}
\newtheorem{aufg}{Aufgabe}

%%%%%%%%%%%%%%%%%%%%%%%%%%%%%%%%%%%%%%%%%%%%%%%%%%%%%%%%%%%%%%%%%%%%%%%%%%%%%%%%

\newenvironment{abc}{\begin{enumerate}\renewcommand{\labelenumi}{\alph{enumi})}}{\end{enumerate}}
\newenvironment{ABC}{\begin{enumerate}\renewcommand{\labelenumi}{\Alph{enumi})}}{\end{enumerate}}
\newenvironment{iii}{\begin{enumerate}\renewcommand{\labelenumi}{(\roman{enumi})}}{\end{enumerate}}
\usepackage{listings}
  \lstset{language=Matlab,
          frame=trbl,
          frameround=tttt,
          basicstyle=\ttfamily,
          breaklines=true,
          escapeinside={"}{"}, % leave listing environment
          showstringspaces=true,
          stringstyle=\RedViolet,
          upquote=true, % to print ' correctly, requires the package textcomp
          columns=flexible, % remove phantom spaces appearing when copying from pdf but destroys fixedwidth :'(
          commentstyle=\OliveGreen\slshape,
          deletekeywords={home, dir},
          keywordstyle=\Blue\bfseries,
          morekeywords={and, blkdiag, case, cat, catch, cd, cell, cell2mat, cell2struct, celldisp, cellplot, char, class, classdef, commandwindow, commandhistory, continue, delete, doc, double, edit, eq, ezplot, factorial, false, fieldnames, filebrowser, fplot, func2str, function_handle, ge, getfield, gcf, givens, int64, isa, iscell, isequal, isfield, islogical, isstruct, isvector, le, logical, ls, mat2cell, methods, mkdir, mldivide, movefile, mrdivide, mtimes, ne, not, num2cell, off, on, ones, or, orderfields, otherwise, parfor, properties, quad, quad2d, repmat, rmdir, rmfield, setfield, single, sortrows, spmd, struct, str2func, struct2cell, switch, true, try, type}}
  \usepackage{textcomp}

% pretty colors
\def\Bittersweet{\color{Bittersweet}}
\def\Black{\color{Black}}
\def\Blue{\color{Blue}}
\def\BlueViolet{\color{BlueViolet}}
\def\CadetBlue{\color{CadetBlue}}
\def\Gray{\color{Gray}}
\def\OliveGreen{\color{OliveGreen}}
\def\lf{\color{OliveGreen}}
\def\Orange{\color{Orange}}
\def\Plum{\color{Plum}}
\def\Purple{\color{Purple}}
\def\Red{\color{Red}}
\def\RedViolet{\color{RedViolet}}
\def\RoyalBlue{\color{RoyalBlue}}
\def\RoyalPurple{\color{RoyalPurple}}
\def\White{\color{White}}
% ------------------------------------------------------------------------------------
% VORTRAG ESCO 2012     FRIEDRICH-ALEXANDER-UNIVERSITAET ERLANGEN-NUERNBERG
% ------------------------------------------------------------------------------------
%     erstellt von Christopher Basting, 21. Juni 2012
% ------------------------------------------------------------------------------------
% abbrev.tex
% ------------------------------------------------------------------------------------

% Mathe spezifische Packages
\usepackage{amsmath,amsfonts}
\usepackage{dsfont}

\renewcommand{\d}{{\rm d}} % d fuer dx bei Ableitungen / Integralen
\newcommand{\gdw}{\Leftrightarrow} % <=>
\newcommand{\x}{\cdot}
\newcommand{\ma}{\left( \begin{array}{*{19}{c}}} % Matrix oeffnen
\newcommand{\me}{\end{array}\right)} % Matrix schliessen
\newcommand{\mA}{\left[ \begin{array}{*{19}{c}}} % Matrix oeffnen
\newcommand{\mE}{\end{array}\right]} % Matrix schliessen
\newcommand{\C}{\mathds{C}}
\newcommand{\E}{\mathds{1}}
\newcommand{\F}{\mathds{F}}
\newcommand{\K}{\mathds{K}}
\newcommand{\N}{\mathds{N}}
\newcommand{\Q}{\mathds{Q}}
\newcommand{\R}{\mathds{R}}
\newcommand{\Z}{\mathds{Z}}
\newcommand{\id}{\mathds{1}}
\renewcommand{\d}{\,{\rm d}}
\newcommand{\D}{{\rm D}}
\newcommand{\richtig}{\checkmark}
\newcommand{\falsch}{\lightning}
\newcommand{\ra}{\rightarrow}
\newcommand{\Ra}{\Rightarrow}
\newcommand{\la}{\leftarrow}
\newcommand{\La}{\Leftarrow}

\newcommand{\abbildung}[3]{#1 \, : \, #2 \rightarrow #3}
\newcommand{\grad}{\nabla}

%\renewcommand{\phi}{\varphi}
\newcommand{\cL}{\mathcal{L}}
\newcommand{\cO}{\mathcal{O}}

\newcommand{\eps}{\varepsilon}

\renewcommand{\vec}{\bold}
\newcommand{\dist}{{\rm dist}}

\newcommand{\mat}[1]{\textsf{\textbf{#1}}}

\usepackage{amscd,amssymb} 

\providecommand{\abs}[1]{\lvert#1\rvert}
\providecommand{\norm}[1]{\left\lVert#1\right\rVert}
\providecommand{\normNoAdjust}[1]{\lVert#1\rVert}
\providecommand{\transpose}[1]{{#1}^\top}
\providecommand{\inversetranspose}[1]{{#1}^{-\top}}



%%%%%%%%%%%%%%%%%%%%%%%%%%%%%%%%%%%%%%%%%%%%%%%%%%%%%%%%%%%%%%%%%%%%%%%%%%%%%%%%

% Customization of captions in floating environments such as figure and table
\usepackage{caption}
\captionsetup{margin=10pt,font=small,labelfont={bf,sf},position=bottom}


\usepackage{lastpage}
\usepackage{fancybox}
\usepackage{comment}
\specialcomment{loesung}{\begingroup\changefont{cmss}{m}{n}\color{fau}}{\endgroup}
\specialcomment{loesungcode}{\begingroup\sffamily}{\endgroup}

%%%%%%%%%%%%%%%%%%%%%%%%%%%%%%%%%%%%%%%%%%%%%%%%%%%%%%%%%%%%%%%%%%%%%%%%%%%%%%%%

\pagestyle{scrheadings}
\rofoot{}
\lofoot{}
\cofoot{}
\lohead{ \includegraphics[height=10mm]{img/tud.pdf}\\[0.5cm]
\changefont{cmss}{m}{n} 
\veranstaltung\\
\dozent\\~\\~
}
\rohead{  \includegraphics[height=10mm]{img/m_fak.pdf} \\[0.5cm]
\changefont{cmss}{m}{n} 
\zeitraum \\ Übungsaufgaben -- Seite \thepage/\pageref*{LastPage} \\[0.2cm] \hrule ~\\[0.2cm]~}
\linespread{1.05}



\excludecomment{loesung}
%\usepackage{draftwatermark} \SetWatermarkText{\textbf{Alle Angaben ohne Gew\"ahr.}} \SetWatermarkLightness{0.85} \SetWatermarkScale{0.3}

\usepackage{lmodern}

% Einige Befehle (s. Thomas Rohkaemper's Slides)
\definecolor{darkgray}{gray}{0.3}
\newcommand{\kommandozeile}[1]{{{\ttfamily\bfseries \color{white}\colorbox{black}{#1}}}}
\newcommand{\dateiname}[1]{{\ttfamily\bfseries \color{TUgreen}#1}}
\newcommand{\emphword}[1]{{\color{TUgreen}#1}}
\newcommand{\keyword}[1]{\texttt{\color{darkgray}#1}}
\newcommand{\emphkeyword}[1]{\texttt{\color{TUgreen}#1}}
\newcommand{\befehl}[1]{\texttt{\color{darkgray}\symbol{`\\}#1}}
\newcommand{\tabitem}{~~\llap{\color{TUgreen}\textbullet}~~}

\usepackage{fancybox, graphicx}

\begin{document}
\changefont{cmss}{m}{n}

%\blattnonum{\"Ubungsaufgaben}

\begin{aufg}[Hello World!]
	Erstellen Sie folgendes "Hello-World" Beispiel.
\begin{verbatim}
\documentclass{article}
\begin{document}
    Hello World!
\end{document}
\end{verbatim}
\end{aufg}

\begin{aufg}[Ein erstes eigenes Dokument] \label{aufgabe-hello-world}
	Erstellen Sie ein einfaches \LaTeX-Dokument. Teilen Sie den Text in Kapitel und Unterkapitel ein. Setzen Sie einige Wörter im Text kursiv, im Fettdruck oder in Schreibmaschinenschrift und in verschiedenen Schriftgrößen. Verwenden Sie zum Beispiel das Paket \keyword{blindtext} zur Erzeugung von Text-Passagen.
\end{aufg}

\begin{aufg}[Umlaute]
	Geben Sie in einem \LaTeX-Dokument die deutschen Umlaute Ä, Ö, Ü, ä, ö, ü, ß aus, indem Sie:
\begin{itemize}
	\item die Umlaute maskieren,
	\item die Umlaute im Quellcode direkt verwenden und den Zeichensatz entsprechend anpassen.
\end{itemize}
\end{aufg}

\begin{aufg}[Zusammensetzen des Dokuments aus mehreren Dateien] \label{aufgabe-mehrere-dateien}
		Erstellen Sie ein Dokument aus den bisherigen Aufgaben, welches aus mehreren Dateien besteht. In der Hauptdatei sollen die Präambel, die Titelseite und das Inhaltsverzeichnis stehen. Jedes Kapitel der obersten Ebene soll in einer eigenen Datei stehen und in die Hauptdatei mittels \befehl{include} eingebunden werden. Hierzu müssen die Dateien der vorherigen Aufgaben in ein neues Verzeichnis kopiert werden und die Befehle \befehl{documentclass} und die \keyword{document}-Umgebung aus den vorherigen Aufgaben entfernt werden, da diese sonst doppelt vorhanden wären.
		 

\end{aufg}

\begin{aufg}[Listen und Aufzählungen]
	Erstellen Sie ein Kochrezept. Verwenden Sie für die Zutatenliste die Aufzählungs-Umgebung \keyword{itemize} und für die Zubereitungshinweise die \keyword{enumerate}-Umgebung.
\end{aufg}

\begin{aufg}[Tabellen]
	\begin{frame}[fragile]
	\frametitle{Tabellen - ein einfaches Beispiel}
	\vspace{-0.9cm}
	\latexBeispielDirekt{Beispiel: einfache Tabelle}{examples/Tabelle_einfach/Tabelle_einfach}
	\vfill
	Tabellenformatierung:\\[0.2cm]
	\begin{tabular}{rlcrl}
		\emphkeyword{l} & linksbündig ausrichten &~~~~~~ & \emphkeyword{|} & einfacher vertikaler Trennstrich \\
		\emphkeyword{c} & zentriert ausrichten && \emphkeyword{||} & doppelter vertikaler Trennstrich\\
		\emphkeyword{r} & rechtsbündig ausrichten && \emphkeyword{@\{text\}} & benutzerdefiniertes Trennzeichen \\
		\emphkeyword{p\{n\}} & Spalte mit fester Breite \keyword{n}
	\end{tabular}
\end{frame}

\begin{frame}
	\frametitle{Tabellen - mehrspaltige Zellen}
	\vspace{-0.9cm}
	\latexBeispielDirekt{Beispiel: mehrspaltige Zellen}{examples/Tabelle_multicolumn/Tabelle_multicolumn}
	\vfill
	Tabellenformatierung:\\[0.2cm]
	\begin{tabular}{rl}
		\emphkeyword{\textbackslash{}hline} & horizontale Linie über die ganze Breite \\
		\emphkeyword{\textbackslash{}vline} & vertikaler Line innerhalb einer Zeile \\
		\emphkeyword{\textbackslash{}cline\{m-n\}} & horizontale Linie von Spalte \keyword{m} bis Spalte \keyword{n}\\
		\emphkeyword{\textbackslash{}multicolumn\{n\}\{format\}\{Inhalt\}} & Zelle über \keyword{n} Spalten
	\end{tabular}
\end{frame}
\end{aufg}

\newpage

\begin{aufg}[Formelsatz]
	Erstellen Sie folgenden Text:
	\begin{center}
		\shadowbox{
			\begin{minipage}{0.9\textwidth}
				Sei $I$ ein reelles Intervall und $f:I\to\mathbb{R}$ eine $(n+1)$-mal stetig differenzierbare Funktion. Dann gilt für alle $a, x\in I$:
				\(f(x) = T_n(x) + R_n(x)\)
				mit dem $n$-ten Taylorpolynom an der Entwicklungsstelle $a$
				\begin{eqnarray*}
				T_(x) & = & \sum_{k=0}^n \frac{f^{(k)}(x)}{k!}(x-a)^k \\
				      & = & f(a) + \frac{f'(a)}{1!} (x-a) + \frac{f''(a)}{2!} (x-a)^2 + \cdots + \frac{f^{(n)}(x)}{n!} (x-a)^n
				\end{eqnarray*}
				und dem $n$-ten Restglied
				\[R_n(x) = \int_a^x \frac{(x-t)^n}{n!} f^{(n+1)}(t)\;\mathrm{d}t\]
				
				In den Formeln stehen $f'$, $f''$, \dots, $f^{(n)}$ für die erste, zweite, ..., $n$-te Ableitung der Funktion $f$.
			\end{minipage}
		}
	\end{center}
	\textit{Hinweis:} Verwenden Sie zur Erzeugung der Mengensymbole $\mathbb{R}$ und $\mathbb{N}$ den im Paket \keyword{amssymb} {enthaltenen} Befehl \befehl{mathbb} und für die ausgerichteten Gleichungen die Umgebung \keyword{align} aus dem \keyword{amsmath} Paket.
\end{aufg}

\begin{aufg}[Inhaltsverzeichnis und Querverweise] \label{aufgabe-querverweise}
	Ändern Sie das Dokument aus Aufgabe 2 so ab, dass eine Titelseite und ein Inhaltsverzeichnis ausgegeben wird. Bauen Sie einige Querverweise mit den Befehlen \befehl{label}, \befehl{ref} und \befehl{pageref} in das Dokument ein.
\end{aufg}

\begin{aufg}[Bilder]
	Binden Sie ein Bild in ein das \LaTeX-Dokument aus Aufgaben 4 ein. Es soll eine Breite von \keyword{6 cm} haben und in einer Figure-Umgebung mit Bezeichnung stehen. Verändern Sie die Positionierung der gleitenden Umgebung \keyword{figure}. Erstellen Sie einen Querverweis auf das Bild.
\end{aufg}

\begin{aufg}[Präsentationen]
	Erstellen Sie eine einfache Präsentation mit mehreren Folien. Benutzen Sie dazu das \LaTeX{} \keyword{beamer}-Paket. Folgende Merkmale sollen in der Präsentation enthalten sein: Titelseite, Aufzählungen, Blöcke, zwei Spalten auf einer Folie und Overlays.
\end{aufg}


\vfill

\hrule
\begin{center}
	Die Präsentation und Übungsaufgaben finden Sie im Moodle-Arbeitsraum der Veranstaltung:\\
	\href{https://moodle.tu-dortmund.de/course/view.php?id=6103}{https://moodle.tu-dortmund.de/course/view.php?id=6103}
\end{center}
%\begin{flushright}
	%\tiny Revision \texttt{\gitAbbrevHash} (\gitAuthorDate)
%\end{flushright}

\end{document}


